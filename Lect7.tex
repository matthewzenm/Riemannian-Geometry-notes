\chapter{Heintze--Karcher Comparison Theorem}

Now we finished our tour in basic classes, and enter the topic classes.
We first go through geometry of submanifolds rapidly, and then provide the Heintze--Karcher comparison theorem.
After this, we give two applications:
Alexandorv's theorem on constan mean curvature (CMC) hypersurface and Levy--Gromov isoperimetric inequality.

\section{Geometry of Submanifolds}

Let $f:\Sigma^k\hookrightarrow(M^n,g)$ be a immersion, equip $\Sigma$ with pullback metric $f^*g$ (still denoted by $g$ for simplicity).
There is a decomposition $\nabla_XY=(\nabla_XY)^\top+(\nabla_XY)^\perp$ for $X,Y\in\mathfrak{X}(\Sigma)$, and simple observation shows
\begin{prop}
    Let $\nabla^\Sigma$ be the Levi--Civita connection on $\Sigma$, then $\nabla^\Sigma$ is given by
    \[\nabla^\Sigma_XY=(\nabla_XY)^\top\]
    for $X,Y\in\mathfrak{X}(\Sigma)$.
\end{prop}

Now we need the conception of normal bundle.
\begin{defn}
    Let $\Sigma$ be a submanifold of $M$, then for any $p\in\Sigma$, we have the decomposition
    \[T_pM=T_p\Sigma+N_p\Sigma,\]
    where $N_p\Sigma$ is the orthogonal complement of $T_p\Sigma$.
    Then define the \textbf{normal bundle} of $\Sigma$ to be
    \[N\Sigma=\bigsqcup_{p\in\Sigma}N_p\Sigma.\]
    It is similar to tangent bundle to give differentiable structure on $N\Sigma$.
    Let $\pi:N\Sigma\to\Sigma$ be the natural projection, we define
    \[\Gamma(N\Sigma)=\{s:\Sigma\to N\Sigma|\ s\text{ smooth and }\pi\circ s=\mathrm{id}\}.\]
\end{defn}

Thus we can define the second fundamental form of a submanifold.
\begin{defn}
    Let $\Sigma$ be a submanifold of $M$, then we define the \textbf{second fundamental form} of $\Sigma$ to be
    \begin{align*}
        \II:\mathfrak{X}(\Sigma)\times\mathfrak{X}(\Sigma)&\to\Gamma(N\Sigma)\\
        (X,Y)&\mapsto(\nabla_XY)^\perp.
    \end{align*}
\end{defn}

Since $\II(X,Y)-\II(Y,X)=(\nabla_XY)^\perp-(\nabla_YX)^\perp=([X,Y])^\perp=0$, the second fundamental form is symmetric.
Moreover, we can define the shape operator.
\begin{defn}
    For a fixed $\xi\in\Gamma(N\Sigma)$, we define
    \begin{align*}
        S_\xi:\mathfrak{X}(\Sigma)&\to\mathfrak{X}(\Sigma)\\
        X&\mapsto\nabla_X\xi-(\nabla_X\xi)^\perp.
    \end{align*}
\end{defn}

We have the following Weingarten formula.
\begin{prop}
    For $\xi\in\Gamma(N\Sigma)$ and $X,Y\in\mathfrak{X}(\Sigma)$, we have
    \[\langle S_\xi(X),Y\rangle=\langle\II(X,Y),-\xi\rangle.\]
    In particular, $S_\xi$ is symmetric.
\end{prop}
\begin{proof}
    We have the calculation
    \begin{align*}
        \langle S_\xi(X),Y\rangle&=\langle\nabla_X\xi-(\nabla_X\xi)^\perp,Y\rangle\\
        &=\langle\nabla_X\xi,Y\rangle-\langle(\nabla_X\xi)^\perp,Y\rangle\\
        &=X\langle\xi,Y\rangle-\langle\xi,\nabla_XY\rangle\\
        &=\langle-\xi,\nabla_XY\rangle=\langle-\xi,(\nabla_XY)^\perp\rangle\\
        &=\langle\II(X,Y),-\xi\rangle.\qedhere
    \end{align*}
\end{proof}

We mention here the Gauss--Codazzi equations.
The proof is pure calculation, so we omit the proof and without having effect on our course.
\begin{thm}
    Let $\Sigma\subset M$ be Riemannian manifolds, then we have\\
    Gauss equation
    \[R^\Sigma(X,Y,Z,W)=R^M(X,Y,Z,W)+\langle\II(X,Z),\II(Y,W)\rangle-\langle\II(X,W),\II(Y,Z)\rangle;\]
    Codazzi equation
    \[R^M(X,Y,Z,\xi)=(\nabla_Y\II)(X,Z,\xi)-(\nabla_X\II)(Y,Z,\xi),\]
    where $\II(X,Y,\xi)=\langle\II(X,Y),\xi\rangle$.
\end{thm}

Lastly, we define the mean curvature of a submanifold.
\begin{defn}
    Let $\Sigma$ be a submanifold of $M$, with second fundamental form $\II$.
    Then the \textbf{mean curvature vector} of $\Sigma$ is defined as
    \[\vec{H}(p)=\tr\II|_p=\sum_{i=1}^k\II(e_i,e_i),\]
    where $\{e_i\}$ is an orthonormal basis at $p\in\Sigma$.

    In particular, if $\Sigma$ is a hypersurface (i.e.\ of codimension $1$) and two-sided (i.e.\ there is a normal unit vector field $N\in\Gamma(N\Sigma)$), then we have
    \[\vec{H}=HN,\ \II=hN,\]
    where $H$ is a function on $\Sigma$, and $h$ is a symmetric $(0,2)$-tensor on $\Sigma$.
\end{defn}

\begin{eg}
    Consider $\mathbb{S}^{n-1}\subset\mathbb{R}^n$.
    Choose the inward unit normal vector field $N(x)=-x$, then we have
    \[\nabla_XN=X(-x)=-X.\]
    Hence we have $h_{ij}=\delta_{ij}$, and $H=n-1$.
\end{eg}

\section{The theorem}

Before we state the theorem, we need some preparation.

Let $\Sigma^{n-1}\subset(M^n,g)$ be an embedded closed hypersurface, with $\Ric^M\geq(n-1)Kg$.
Let $\Omega$ be a region enclosed by $\Sigma$, and $N$ be an inward unit normal vector field on $\Sigma$.
Consider for $p\in\Sigma$, let $\gamma_{N(p)}(t)=\exp_p(tN(p))$.
Define
\[\tau(p)=\sup\{t\in(0,+\infty)|\ d(\gamma_{N(p)},\Sigma)=t\},\]
then Kasue showed in 1980s that
\[\Cut(\Sigma)=\{q\in M|\ q=\exp_p(\tau(p)N(p))\}\]
is of zero-measure.

Then we state the theorem.
\begin{thm}[Heintze--Karcher]
    Let $(M^n,g)$ be a Riemannian manifold with $\Ric\geq(n-1)Kg$.
    Let $\Sigma\subset M$ be an embedded closed hypersurface that encloses $\Omega$.
    Then
    \[\Vol(\Omega)\leq\int_\Sigma\int_0^{\tau(p)}\left(\sn'_K(r)-\frac{H(p)}{n-1}\sn_K(r)\right)^{n-1}\d{r}\d\Vol_{\Sigma}(p),\]
    where
    \[\sn_K(r)=\begin{cases}
        \frac{1}{\sqrt{K}}\sin(\sqrt{K}r), & K>0,\\
        r, & K=0,\\
        \frac{1}{\sqrt{-K}}\sinh(\sqrt{-K}r), & K<0.
    \end{cases}\]
\end{thm}
\begin{proof}
    Under Fermi coordinate $(q,r)\mapsto\exp_p(rN(p))$, we have the metric $g=\d{r}^2+g_{\Sigma}$, and the volume form of $M$ can be written
    \[\d\Vol_g(p,r)=\mathscr{J}(p,r)\d{r}\d\Vol_{\Sigma}(p).\]
    Let $d_\Sigma$ be the distance function to $\Sigma$, then one can show similarly to Proposition~\ref{Laplace of distance function}~that
    \[\Delta d_\Sigma(p,r)=\frac{\partial{}}{\partial{r}}\log\mathscr{J}(p,r).\]
    Denote $\Phi=\mathscr{J}^{\frac{1}{n-1}}$, similarly there holds $\frac{\partial^2{}}{\partial{r^2}}\Phi(p,r)+K\Phi(p,r)\leq 0$ by $\Ric\geq(n-1)Kg$ and Bochner formula.
    We need to calculate $\frac{\partial{}}{\partial{r}}\Phi(p,0)$.
    We know that $\Phi(p,0)=1$, then we have (we omit $p$)
    \begin{align*}
        \Phi'(0)&=\frac{1}{n-1}\mathscr{J}^{\frac{2-n}{n-1}}(0)\mathscr{J}'(0)\\
        &=\frac{1}{n-1}\mathscr{J}'(0)\\
        &=\frac{1}{n-1}\left.\frac{\partial{}}{\partial{r}}\right|_{r=0}\sqrt{\det((g_\Sigma)_{ij})}\\
        &=\frac{1}{n-1}\left(\frac{1}{2}\cdot\frac{1}{\sqrt{\det((g_r)_{ij})}}\det((g_r)_{ij})g^{kl}(\langle\nabla_{\partial r}\partial_l,\partial_k\rangle+\langle\partial_k,\nabla_{\partial r}\partial_l\rangle)\right)\\
        &=\frac{1}{n-1}\left(\mathscr{J}(0)g^{kl}\langle\nabla_{\partial r}\partial_l,\partial_k\rangle\right)\\
        &=\frac{1}{n-1}\left(g^{kl}\langle\nabla_{\partial l}\partial_r,\partial_k\rangle\right)\\
        &=\frac{1}{n-1}\left(g^{kl}\langle S_{\partial r}(\partial l),\partial_k\rangle\right)\\
        &=\frac{1}{n-1}\left(g^{kl}(-h_{lk})\right)\\
        &=-\frac{1}{n-1}\tr{h}=-\frac{H(p)}{n-1}.
    \end{align*}
    Then let $\Phi_K$ satisfy $\frac{\partial^2{}}{\partial{r^2}}\Phi_K(p,r)+K\Phi_K(p,r)=0$, $\Phi_K(p,0)=0$, $\Phi_K'(p,0)=-\frac{1}{n-1}H(p)$.
    Then we have
    \[\Phi_K(p,r)=\sn_K'(r)-\frac{1}{n-1}H(p)\sn_K(r),\]
    and the theorem follows from Sturm--Liouville comparison theorem.
\end{proof}

\begin{rem}
    \begin{enumerate}[(1)]
        \item More precise estimation can show that
        \[\Vol(\Omega)\leq\int_\Sigma\int_0^{\tau(p)}\prod_{i=1}^{n-1}(\sn'_K(r)-\kappa_i(p)\sn_K(r))\d{r}\d\Vol_\Sigma(p),\]
        where $\kappa_i$ are the principal curvatures.
        This implies Heintze--Karcher's estimate by AM-GM inequality.
        \item $\Phi_K^{n-1}(p,r)\d{r}\d\Vol_\Sigma$ cannot be regarded as volume form of $M_K$.
    \end{enumerate}
\end{rem}

\section{Alexandorv's Theorem}

The first application of Heintze--Karcher volume estimate is Alexandorv's theorem on CMC hypersurface.

Before we start to discuss the theorem, we first need to discuss the first variation formula of area functional (we use area to denote the volume of a hypersurface), in order to introduce the notion of minimal surfaces.

\begin{prop}[First variation formula for area]
    Let $\varphi:\Sigma^k\hookrightarrow(M^n,g)$ is an isometric immersion.
    Let area functional be
    \[A(\Sigma^k)=\int_{\Sigma^k}\d{A_{\varphi^*g}}.\]
    Let $f:\Sigma\times(-\varepsilon,\varepsilon)\to M$ be a family of immersions, denote $f(\Sigma,t)=\Sigma_t$.
    Let $X:=\left.\frac{\partial{}}{\partial{t}}\right|_{t=0}f\in TM$ be the variation vector field, and suppose $\operatorname{supp}X$ is compact.
    Then we have
    \[\left.\frac{\d{}}{\d{t}}\right|_{t=0}A(\Sigma_t)=-\int_\Sigma\langle\vec{H},X\rangle\d{A_\Sigma}.\]
\end{prop}
\begin{proof}
    Let $f^*_tg=g_t$.
    Then we have
\end{proof}