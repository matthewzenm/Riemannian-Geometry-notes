\chapter{Heintze--Karcher Comparison Theorem}

Now we finished our tour in basic classes, and enter the topic classes.
We first go through geometry of submanifolds rapidly, and then provide the Heintze--Karcher comparison theorem.
After this, we give two applications:
Alexandorv's theorem on constan mean curvature (CMC) hypersurface and Levy--Gromov isoperimetric inequality.

\section{Geometry of Submanifolds}

Let $f:\Sigma^k\hookrightarrow(M^n,g)$ be an immersion, equip $\Sigma$ with pullback metric $f^*g$ (still denoted by $g$ for simplicity).
There is a decomposition $\nabla_XY=(\nabla_XY)^\top+(\nabla_XY)^\perp$ for $X,Y\in\mathfrak{X}(\Sigma)$, and simple observation shows
\begin{prop}\index{Levi--Civita connection!of submanifold}
    Let $\nabla^\Sigma$ be the Levi--Civita connection on $\Sigma$, then $\nabla^\Sigma$ is given by
    \[\nabla^\Sigma_XY=(\nabla_XY)^\top\]
    for $X,Y\in\mathfrak{X}(\Sigma)$.
\end{prop}

Now we need the conception of normal bundle.
\begin{defn}\index{normal bundle}
    Let $\Sigma$ be a submanifold of $M$, then for any $p\in\Sigma$, we have the decomposition
    \[T_pM=T_p\Sigma+N_p\Sigma,\]
    where $N_p\Sigma$ is the orthogonal complement of $T_p\Sigma$.
    Then define the \textbf{normal bundle} of $\Sigma$ to be
    \[N\Sigma=\bigsqcup_{p\in\Sigma}N_p\Sigma.\]
    It is similar to tangent bundle to give differentiable structure on $N\Sigma$.
    Let $\pi:N\Sigma\to\Sigma$ be the natural projection, we define
    \[\Gamma(N\Sigma)=\{s:\Sigma\to N\Sigma|\ s\text{ smooth and }\pi\circ s=\mathrm{id}\}.\]
\end{defn}

Thus we can define the second fundamental form of a submanifold.
\begin{defn}\index{second fundamental form}
    Let $\Sigma$ be a submanifold of $M$, then we define the \textbf{second fundamental form} of $\Sigma$ to be
    \begin{align*}
        \II:\mathfrak{X}(\Sigma)\times\mathfrak{X}(\Sigma)&\to\Gamma(N\Sigma)\\
        (X,Y)&\mapsto(\nabla_XY)^\perp.
    \end{align*}
\end{defn}

Since $\II(X,Y)-\II(Y,X)=(\nabla_XY)^\perp-(\nabla_YX)^\perp=([X,Y])^\perp=0$, the second fundamental form is symmetric.
Moreover, we can define the shape operator.
\begin{defn}\index{shape operator}
    For a fixed $\xi\in\Gamma(N\Sigma)$, we define
    \begin{align*}
        S_\xi:\mathfrak{X}(\Sigma)&\to\mathfrak{X}(\Sigma)\\
        X&\mapsto\nabla_X\xi-(\nabla_X\xi)^\perp.
    \end{align*}
\end{defn}

We have the following Weingarten formula.
\begin{prop}\index{Weingarten formula}
    For $\xi\in\Gamma(N\Sigma)$ and $X,Y\in\mathfrak{X}(\Sigma)$, we have
    \[\langle S_\xi(X),Y\rangle=\langle\II(X,Y),-\xi\rangle.\]
    In particular, $S_\xi$ is symmetric.
\end{prop}
\begin{proof}
    We have the calculation
    \begin{align*}
        \langle S_\xi(X),Y\rangle&=\langle\nabla_X\xi-(\nabla_X\xi)^\perp,Y\rangle\\
        &=\langle\nabla_X\xi,Y\rangle-\langle(\nabla_X\xi)^\perp,Y\rangle\\
        &=X\langle\xi,Y\rangle-\langle\xi,\nabla_XY\rangle\\
        &=\langle-\xi,\nabla_XY\rangle=\langle-\xi,(\nabla_XY)^\perp\rangle\\
        &=\langle\II(X,Y),-\xi\rangle.\qedhere
    \end{align*}
\end{proof}

We mention here the Gauss--Codazzi equations.
The proof is pure calculation, so we omit the proof and without having effect on our course.
\begin{thm}\index{theorem!Gauss--Codazzi}
    Let $\Sigma\subset M$ be Riemannian manifolds, then we have\\
    Gauss equation
    \[R^\Sigma(X,Y,Z,W)=R^M(X,Y,Z,W)+\langle\II(X,Z),\II(Y,W)\rangle-\langle\II(X,W),\II(Y,Z)\rangle;\]
    Codazzi equation
    \[R^M(X,Y,Z,\xi)=(\nabla_Y\II)(X,Z,\xi)-(\nabla_X\II)(Y,Z,\xi),\]
    where $\II(X,Y,\xi)=\langle\II(X,Y),\xi\rangle$.
\end{thm}

Lastly, we define the mean curvature of a submanifold.
\begin{defn}\index{mean curvature}
    Let $\Sigma$ be a submanifold of $M$, with second fundamental form $\II$.
    Then the \textbf{mean curvature vector} of $\Sigma$ is defined as
    \[\vec{H}(p)=\tr\II|_p=\sum_{i=1}^k\II(e_i,e_i),\]
    where $\{e_i\}$ is an orthonormal basis at $p\in\Sigma$.

    In particular, if $\Sigma$ is a hypersurface (i.e.\ of codimension $1$) and two-sided (i.e.\ there is a normal unit vector field $N\in\Gamma(N\Sigma)$), then we have
    \[\vec{H}=HN,\ \II=hN,\]
    where $H$ is a function on $\Sigma$, and $h$ is a symmetric $(0,2)$-tensor on $\Sigma$.
\end{defn}

\begin{eg}
    Consider $\mathbb{S}^{n-1}\subset\mathbb{R}^n$.
    Choose the inward unit normal vector field $N(x)=-x$, then we have
    \[\nabla_XN=X(-x)=-X.\]
    Hence we have $h_{ij}=\delta_{ij}$, and $H=n-1$.
\end{eg}

\begin{defn}\index{principal curvature}
    Let $\Sigma\subset M$ be an two-sided hypersurface, $N$ be a unit normal vector field of $\Sigma$.
    Then the eigenvalues of $S_N$ are called \textbf{principal curvatures}.
\end{defn}

\section{The theorem}

Before we state the theorem, we need some preparation.

Let $\Sigma^{n-1}\subset(M^n,g)$ be an embedded closed hypersurface, with $\Ric^M\geq(n-1)Kg$.
Let $\Omega$ be a region enclosed by $\Sigma$, and $N$ be an inward unit normal vector field on $\Sigma$.
Consider for $p\in\Sigma$, let $\gamma_{N(p)}(t)=\exp_p(tN(p))$.
Define
\[\tau(p)=\sup\{t\in(0,+\infty)|\ d(\gamma_{N(p)},\Sigma)=t\},\]
then Kasue showed in 1980s that
\[\Cut(\Sigma)=\{q\in M|\ q=\exp_p(\tau(p)N(p))\}\]
is of zero-measure.

Then we state the theorem.
\begin{thm}[Heintze--Karcher]\index{theorem!Heintze--Karcher}
    Let $(M^n,g)$ be a Riemannian manifold with $\Ric\geq(n-1)Kg$.
    Let $\Sigma\subset M$ be an embedded closed hypersurface that encloses $\Omega$.
    Then
    \[\Vol(\Omega)\leq\int_\Sigma\int_0^{\tau(p)}\left(\sn'_K(r)-\frac{H(p)}{n-1}\sn_K(r)\right)^{n-1}\d{r}\d\Vol_{\Sigma}(p),\]
    where
    \[\sn_K(r)=\begin{cases}
        \frac{1}{\sqrt{K}}\sin(\sqrt{K}r), & K>0,\\
        r, & K=0,\\
        \frac{1}{\sqrt{-K}}\sinh(\sqrt{-K}r), & K<0.
    \end{cases}\]
    The equality holds if and only if $\partial\Omega$ is umbilical.
\end{thm}
\begin{proof}
    Under Fermi coordinate $(q,r)\mapsto\exp_p(rN(p))$, we have the metric $g=\d{r}^2+g_{\Sigma}$, and the volume form of $M$ can be written
    \[\d\Vol_g(p,r)=\mathscr{J}(p,r)\d{r}\d\Vol_{\Sigma}(p).\]
    Let $d_\Sigma$ be the distance function to $\Sigma$, then one can show similarly to Proposition~\ref{Laplace of distance function}~that
    \[\Delta d_\Sigma(p,r)=\frac{\partial{}}{\partial{r}}\log\mathscr{J}(p,r).\]
    Denote $\Phi=\mathscr{J}^{\frac{1}{n-1}}$, similarly there holds $\frac{\partial^2{}}{\partial{r^2}}\Phi(p,r)+K\Phi(p,r)\leq 0$ by $\Ric\geq(n-1)Kg$ and Bochner formula.
    We need to calculate $\frac{\partial{}}{\partial{r}}\Phi(p,0)$.
    We know that $\Phi(p,0)=1$, then we have (we omit $p$)
    \begin{align*}
        \Phi'(0)&=\frac{1}{n-1}\mathscr{J}^{\frac{2-n}{n-1}}(0)\mathscr{J}'(0)\\
        &=\frac{1}{n-1}\mathscr{J}'(0)\\
        &=\frac{1}{n-1}\left.\frac{\partial{}}{\partial{r}}\right|_{r=0}\sqrt{\det((g_\Sigma)_{ij})}\\
        &=\frac{1}{n-1}\left(\frac{1}{2}\cdot\frac{1}{\sqrt{\det((g_r)_{ij})}}\det((g_r)_{ij})g^{kl}(\langle\nabla_{\partial r}\partial_l,\partial_k\rangle+\langle\partial_k,\nabla_{\partial r}\partial_l\rangle)\right)\\
        &=\frac{1}{n-1}\left(\mathscr{J}(0)g^{kl}\langle\nabla_{\partial r}\partial_l,\partial_k\rangle\right)\\
        &=\frac{1}{n-1}\left(g^{kl}\langle\nabla_{\partial l}\partial_r,\partial_k\rangle\right)\\
        &=\frac{1}{n-1}\left(g^{kl}\langle S_{\partial r}(\partial l),\partial_k\rangle\right)\\
        &=\frac{1}{n-1}\left(g^{kl}(-h_{lk})\right)\\
        &=-\frac{1}{n-1}\tr{h}=-\frac{H(p)}{n-1}.
    \end{align*}
    Then let $\Phi_K$ satisfy $\frac{\partial^2{}}{\partial{r^2}}\Phi_K(p,r)+K\Phi_K(p,r)=0$, $\Phi_K(p,0)=1$, $\Phi_K'(p,0)=-\frac{1}{n-1}H(p)$.
    Then we have
    \[\Phi_K(p,r)=\sn_K'(r)-\frac{1}{n-1}H(p)\sn_K(r),\]
    and the theorem follows from Sturm--Liouville comparison theorem.
    If the equality holds, we trace back where $\ddot{\Phi}+K\Phi=0$ holds.
    This is in the argument at Bochner formula part which asserts that $\nabla^2d_\Sigma(p,r)$ has equal eigenvalues.
    However, since musical isomorphism commutes with covariant derivative, we just need to check bilinear form
    \[\langle\nabla_X(\nabla d_\Sigma),Y\rangle=\langle\nabla_X\partial_r,Y\rangle=\langle S_{\partial_r}(X),Y\rangle=-h(X,Y)\]
    for $X,Y\in\mathfrak{X}(\Sigma)$.
    That is, the eigenvalues of second fundamental form of $\partial\Omega$ is equal.
    Hence $\partial\Omega$ is umbilical.
\end{proof}

\begin{rem}
    \begin{enumerate}[(1)]
        \item More precise estimation can show that
        \[\Vol(\Omega)\leq\int_\Sigma\int_0^{\tau(p)}\prod_{i=1}^{n-1}(\sn'_K(r)-\kappa_i(p)\sn_K(r))\d{r}\d\Vol_\Sigma(p),\]
        where $\kappa_i$ are the principal curvatures.
        This implies Heintze--Karcher's estimate by AM-GM inequality.
        \item $\Phi_K^{n-1}(p,r)\d{r}\d\Vol_\Sigma$ cannot be regarded as volume form of $M_K$.
    \end{enumerate}
\end{rem}

\section{Alexandorv's Theorem}

The first application of Heintze--Karcher volume estimate is Alexandorv's theorem on CMC hypersurface.

Before we start to discuss the theorem, we first need to discuss the first variation formula of area functional (we use area to denote the volume of a submanifold), in order to introduce the notion of minimal submanifolds.

\begin{prop}[First variation formula for area]\index{first variation formula!for area}
    \footnote{This version is different from which on class. We introduce the notion of proper variation, instead of $\operatorname{supp}X$ being compact.}
    Let $\varphi:\Sigma^k\hookrightarrow(M^n,g)$ is an isometric immersion.
    Let area functional be
    \[A(\Sigma^k)=\int_{\Sigma^k}\d{A_{\varphi^*g}}.\]
    Let $f:\Sigma\times(-\varepsilon,\varepsilon)\to M$ be a family of immersions, denote $f(\Sigma,t)=\Sigma_t$, and suppose $f(p,0)=p$ for all $p\in M$.
    Let $X:=\left.\frac{\partial{}}{\partial{t}}\right|_{t=0}f\in TM$ be the variation vector field.
    Then we have
    \[\left.\frac{\d{}}{\d{t}}\right|_{t=0}A(\Sigma_t)=-\int_\Sigma\langle\vec{H},X\rangle\d{A_\Sigma}.\]
\end{prop}
\begin{proof}
    Let $f^*_tg=g_t$, and
    \[\mathscr{J}(p,t)=\frac{\sqrt{\det{g_t}}}{\sqrt{\det{g_0}}}.\]
    Then we have
    \[A_t(p)=\mathscr{J}(p,t)A_0(x),\]
    so we just need to calculate $\partial_t\mathscr{J}(p,t)$ at $t=0$.
    We have (omit $p$)
    \begin{align*}
        \left.\frac{\partial{}}{\partial{t}}\right|_{t=0}\mathscr{J}(t)&=\frac{1}{\sqrt{\det{g_0}}}\left.\frac{\partial}{\partial{t}}\right|_{t=0}\sqrt{\det{g_t}}\\
        &=\frac{1}{\det{g_0}}\cdot\frac{1}{2}\cdot\sqrt{\det{g_0}}\cdot g^{ij}_0\left.\frac{\partial{}}{\partial{t}}\right|_{t=0}g_{ij}\\
        &=\frac{1}{2}g_0^{ij}\left(\left\langle\nabla_{\partial_t}\frac{\partial{f}}{\partial{x_i}},\frac{\partial{f}}{\partial{x_j}}\right\rangle+\left\langle\frac{\partial{f}}{\partial{x_i}},\nabla_{\partial_t}\frac{\partial{f}}{\partial{x_j}}\right\rangle\right)\\
        &=g^{ij}_0\left\langle\nabla_{\partial_i}\frac{\partial{f}}{\partial{t}},\frac{\partial{f}}{\partial{x^j}}\right\rangle\\
        &=g^{ij}_0\left\langle S_{X}\left(\frac{\partial{f}}{\partial{x^i}}\right),\frac{\partial{f}}{\partial{x^j}}\right\rangle\\
        &=-\left\langle g^{ij}_0\II\left(\frac{\partial{f}}{\partial{x^i}},\frac{\partial{f}}{\partial{x^j}}\right),X\right\rangle\\
        &=-\langle \vec{H},X\rangle.
    \end{align*}
    Take integral we obtain the required formula.
\end{proof}

\begin{cor}
    If $\Sigma\subset M$ minimizes area in any small variation, then $\Sigma$ has mean curvature $0$.
\end{cor}

Thus we can have the notion of minimal submanifold.
\begin{defn}\index{minimal submanifold}
    Let $\Sigma$ be a submanifold of $M$, if $\Sigma$ has constant mean curvature $0$, then $\Sigma$ is called a \textbf{minimal submanifold}.
\end{defn}

\begin{rem}
    We do not demand a minimal submanifold actually minimizes area.
\end{rem}

Now we state Alexandorv's theorem.
\begin{thm}[Alexandorv]\index{theorem!Alexandorv}
    Any embedded closed constant mean curvature (CMC) hypersurface in $\mathbb{R}^n$ is a round sphere.
\end{thm}

\begin{rem}
    The original proof of Alexandorv in 1958 used the celebrated \emph{moving plane method} in theory of PDE.
    However, we will use a geometric approach to prove the theorem, following the work of Montiel and Ros in 1991.
\end{rem}

Before we prove Alexandorv's theorem, we need Ros' theorem and Minkowski's formula as preparation.

\begin{thm}[Ros]\index{theorem!Ros}
    Let $(M,g)$ be a compact Riemannian manifold with $\Ric\geq 0$.
    Let $\Sigma=\partial M$, and assume $\Sigma$ is a CMC submanifold with $H_\Sigma>0$.
    Then
    \[\Vol(M)\leq\frac{n}{n+1}\int_\Sigma\frac{1}{H}\d{A}.\]
    The equality holds if and only if $\Sigma$ is umbilical.
\end{thm}
\begin{proof}
    In the proof of Heintze--Karcher comparison theorem, we can compare $M$ to Euclidean space and obtain
    \[0\leq\Phi(r)\leq 1-\frac{H(p)}{n-1}r,\]
    where $r$ equals to the distance function $d_p$.
    This implies
    \[r\leq\frac{n-1}{H(p)}.\]
    Hence by Heintze--Karcher comparison theorem again, we have
    \begin{align*}
        \Vol(M)&\leq\int_\Sigma\int_0^{\tau(p)}\left(1-\frac{H(p)}{n-1}r\right)^{n-1}\d{r}\d\Vol_{\Sigma}(p)\\
        &\leq\int_\Sigma\int_{0}^{(n-1)/H(p)}\left(1-\frac{H(p)}{n-1}r\right)^{n-1}\d{r}\d\Vol_{\Sigma}(p)\\
        &=\int_\Sigma\frac{n}{n+1}\cdot\frac{1}{H(p)}\d\Vol_\Sigma(p).
    \end{align*}
    The equality holds if and only if the equality in Heintze--Karcher theorem holds, that is, $\Sigma$ is umbilical. 
\end{proof}

\begin{thm}[Minkowski's formula]\index{Minkowski's formula}
    Let $\Sigma^{n-1}\hookrightarrow\mathbb{R}^n$ be a closed hypersurface. Then
    \[\int_\Sigma(H\langle x,N\rangle+n-1)\d{A}=0,\]
    where $x$ is the position function, and $N$ is the inward unit normal vector field.
\end{thm}
\begin{proof}
    First, we have
    \[\div\nolimits_\Sigma x=\sum_{i=1}^{n-1}\langle(\nabla_{e_i}x)^\top,e_i\rangle=n-1,\]
    where $\{e_i\}$ is an orthonormal basis of $T_p\Sigma$, and equality holds since $\nabla_{Y}x=Y$ for any $Y\in T_p\mathbb{R}^n$.
    Moreover, we have
    \begin{align*}
        \div\nolimits_\Sigma(x^\top)&=\div\nolimits_\Sigma(x)-\div\nolimits_\Sigma(x^\perp)\\
        &=n-1+\sum_{i=1}^{n-1}\langle\nabla_{e_i}x^\perp,e_i\rangle\\
        &=n-1+\sum_{i=1}^{n-1}\langle S_{x^\perp}(e_i),e_i\rangle\\
        &=n-1+\sum_{i=1}^{n-1}\langle\II(e_i,e_i),-x^\perp\rangle\\
        &=n-1+H\langle x,N\rangle.
    \end{align*}
    Then by Stokes formula, we have
    \[0=\int_\Sigma\div\nolimits_\Sigma(x^\top)=\int_\Sigma(n-1+H\langle x,N\rangle)\d{A}.\qedhere\]
\end{proof}

\begin{proof}[Proof of Alexandorv's theorem]
    Let $\Sigma$ be a closed CMC hypersurface in $\mathbb{R}^n$, and the region $\Omega$ satisfy $\partial\Omega=\Sigma$.
    We first show if $H(p)$ is constant, then $H(p)>0$.
    Since $\Sigma$ is closed, it is enclosed by a sufficiently large sphere.
    Shrink the sphere until it is tangent to $\Sigma$ at a point.
    Then at the point, every principal eigenvalue of $\Sigma$ is greater than the sphere, hence is positive.
    Therefore $H(p)$ is positive on $\Sigma$.

    Now we complete the proof.
    We have
    \begin{align*}
        nH\Vol(\Omega)&=H\int_\Omega n &\\
        &=H\int_\Omega\div\nolimits_{\mathbb{R}^n}x &\\
        &=-H\int_\Sigma\langle x,N\rangle & \text{(Divergence theorem)}\\
        &=(n-1)|\Sigma|. & \text{(Minkowski's formula)}
    \end{align*}
    Then $\Vol(\Omega)=\frac{n-1}{n}\int_\Sigma\frac{1}{H}$.
    This means the equality in Ros' theorem holds, then $\Sigma$ is umbilical in $\mathbb{R}^n$.
    Hence $\Sigma$ must be a round sphere.
\end{proof}

\begin{rem}
    If $\Sigma\hookrightarrow\mathbb{R}^3$ is an immersion, then there exists closed CMC immersion which is not a round shpere.
    A first counterexample was given by Wente in 1986, called Wente's torus.
    Kapouleas constructed CMC immersions for any topology in 1990.
\end{rem}

\section{Levy--Gromov Isoperimetric Inequality}

We introduce another application of Heintze--Karcher volume estimate, which is the isoperimetric inequality of Levy and Gromov.

We first recall isoperimetric problem.
In $\mathbb{R}^n$, among $C^1$ bounded domain with fixed volume, round ball minimizes the boundary area.
Schmidt proved in 1930s, that in $\mathbb{S}^n$ the same result holds.

First we need a conception.
\begin{defn}
    Let $(M^n,g)$ be a Riemannian manifold, $v\in(0,\Vol(M))$.
    Then we define the \textbf{isoperimetric profile} to be
    \[I_M(v)=\inf\{A(\partial\Omega)|\ \Omega\subset M^n,\ \Vol(\Omega)=v\}.\]
\end{defn}
\begin{eg}
    The isoperimetric profile of $\mathbb{R}^n$ is easy, one can calculate
    \[I_{\mathbb{R}^n}(v)=n|B_1|^{1/n}v^{(n-1)/n}.\]
\end{eg}

By abuse of notation, we have the following geometric quantity:
Take $\beta\in(0,1]$, consider $B(\beta)$ be a geodesic ball on $\mathbb{S}^n$ such that $\Vol(B(\beta))=\beta\Vol(\mathbb{S}^n)$.
Then we denote
\[I(\beta)=\frac{A(\partial B(\beta))}{\Vol(\mathbb{S}^n)}.\]

Now we can state Levy and Gromov's isometric inequality.
\begin{thm}[Levy--Gromov]\label{Levy Gromov}\index{theorem!Levy--Gromov}
    Let $(M^n,g)$ be a Riemannian manifold with $\Ric\geq(n-1)Kg>0$.
    Let $\Omega\subset M$ be a bounded region such that $\Vol(\Omega)=\beta\Vol(M)$.
    Then
    \[\frac{A(\partial\Omega)}{\Vol(M)}\geq I(\beta)=\frac{A(\partial B(\beta))}{\Vol(\mathbb{S}^n)}.\]
    That is, $I_M(\beta)\geq I(\beta)$.
    The equality holds if and only if $M$ is isometric to $\mathbb{S}^n(1/\sqrt{K})$.
\end{thm}

\begin{rem}\label{rem under L-G}
    Before proving the theorem, we need some remarks.
    Let $\Omega$ be an isoperimetric region, that is, $\Omega$ minimizes $A(\partial\Omega)$.
    We want to ask its existence and regularity problems.
    Firstly, since $M$ is positively curved, Bonnet--Myers theorem implies $M$ is compact, and calculus of variation shows isoperimetric region exists.
    Secondly, geometric measure theory shows $H_{\partial\Omega}$ is constant for regular points.
    These discussions are far beyond our scope, so we just mention them here.
\end{rem}

\begin{proof}[Proof of Theorem~\ref{Levy Gromov}]
    As usual, we scale the metric to let $K=1$.
    Compare $M$ to $M_1$, we have
    \[0\leq\Phi(r)\leq\cos{r}-\frac{H_0}{n-1}\sin{r}.\]
    Hence the distance function must satisfy
    \[\tau(p)\leq\cot^{-1}\left(\frac{H_0}{n-1}\right)=:r_0.\]
    Then by Heintze--Karcher volume estimate, we have
    \[\Vol(\Omega)\leq\int_{\partial\Omega}\int_0^{r_0}\left(\cos{r}-\frac{H_0}{n-1}\sin{r}\right)^{n-1}\d{r}\d\Vol_{\partial\Omega}.\]
    The integrand of outer integral is a constant by Remark~\ref{rem under L-G}, hence we have
    \begin{align*}
        \Vol(\Omega)&\leq A(\partial\Omega)\int_{0}^{r_0}\left(\cos{r}-\frac{H_0}{n-1}\sin{r}\right)^{n-1}\d{r}\\
        &=A(\partial\Omega)\frac{\Vol(B_{r_0})}{A(\partial B_{r_0})},
    \end{align*}
    where $B_{r_0}$ is the geodesic ball of radius $r_0$ in $\mathbb{S}^n$.
    Set $a(r_0)=\frac{A(\partial B_{r_0})}{\Vol(B_{r_0})}$, the above inequality can be written
    \[\frac{A(\partial\Omega)}{\Vol(M)}\geq\beta a(r_0).\]
    Same work on $\Omega^c$ shows
    \[\frac{A(\partial\Omega)}{\Vol(M)}\geq(1-\beta)a(\pi-r_0).\]
    Hence
    \[\frac{A(\partial\Omega)}{\Vol(M)}\geq\inf\{\beta a(r_0),(1-\beta)a(\pi-r_0)\},\]
    and monotonicity shows the infimum is attained when $\beta a(t_0)=(1-\beta)a(\pi-t_0)$ for some $t_0$, and in this situation we can calculate that $\beta a(r_0)=I(\beta)$.
    Hence we have the inequality
    \[\frac{A(\partial\Omega)}{\Vol(M)}\geq I(\beta).\]
    If the equality holds, then the equality holds in Heintze--Karcher volume estimate, this shows $M$ is isometric to $\mathbb{S}^n$.
\end{proof}

\begin{rem}
    Bayle gave an alternative proof in his doctoral thesis in 2004.
    He uses the function
    \[f(v)=I^{\frac{n}{n-1}}_M(v)\]
    and uses second variation formula for area functional to obtain
    \[f''(v)\leq nf(v)^{-\frac{n-2}{n}}.\]
    Moreover, if $M=\mathbb{S}^n$, then the equality holds.
    Thus one can use comparison theorem in ODE theory to obtain the result.

    Klatag (2017) applies this method to metric measure space $\mathrm{RCD}(K,n)$ using optimal transport.
\end{rem}

As a finale, we quote some remarkable works of Brendle.

\begin{thm}[Brendle]\index{theorem!Brendle}
    If $M$ is a noncompact, complete Riemannian manifold with $\Ric\geq 0$.
    Assume $M$ has Euclidean volume growth.
    Let $\Omega\subset M$ bounded, then
    \[A(\partial\Omega)^{\frac{n}{n-1}}\geq n|B_1|^{1/n}\Vol(\Omega)^{\frac{n-1}{n}}\theta^{1/n},\]
    where
    \[\theta=\lim_{r\to\infty}\frac{\Vol(B_r)}{|B_1|r^n}\]
    is the asymptotic volume ratio of $M$.
    The equality holds if and only if $\Omega\subset\mathbb{R}^n$.
\end{thm}

Cabre used ABP (Alexandorv--Bakelman--Pucci) method to prove isoperimetric inequality in $\mathbb{R}^n$, Brendle generalized Cabre's proof to the following theorem.

\begin{thm}[Brendle]\index{theorem!Brendle}
    Let $\Sigma^n\subset\mathbb{R}^{n+k}$ be a minimal submanifold.
    Then
    \[|\partial\Sigma|\geq n|B_1|^{1/n}\Vol(\Sigma)^{\frac{n-1}{n}}\]
    when $k\leq 2$.
    The equality holds if and only if $\Sigma\cong\mathbb{B}^n\subset\mathbb{R}^n$.
\end{thm}

Our last theorem is

\begin{thm}[Brendle]\index{theorem!Brendle}
    We have
    \[\int_\Sigma H+|\partial\Sigma|\geq n|B_1|^{1/n}|\Sigma|^\frac{n-1}{n}.\]
\end{thm}

Please refer to \emph{The Isoperimetric Inequality} (Brendle--Eichmair, 2024) for detailed discussion.