\chapter{Preface}

This note is the course note for the Riemannian geometry course in 2024 BICMR summer school on differential geometry, lectured by Prof.\ Xia Chao from Xiamen University.
This course contains basics in Riemannian geometry and advance topics such as Heintz--Karcher comparison theorem and Berstein problem.
Since the course only lasted for 10 days, many aspects of Riemannian geometry is not covered.
For example, the proofs of Hopf--Rinow theorem and Cartan--Hadamard theorem are not given in class, classification of space forms is not mentioned, the progress of Bernstein promblem is only summarized, and so on.
However, this course is enough to be an introduction to Riemannian geometry, at least it is enough for the author himself to study further geometry.

This note has memorable significance to me.
For the study of geometry, this course is the first time that I formally study Riemannian geometry, and I rapidly developed my knowledge of geometry after participating the summer school.
For the note itself, this is the first complete course note of mine, and my \LaTeX{} style has finalized after typing this note.

The summer school has ended for half a year, and I suddenly began to revise this note.
This is because I'm going to use this note as a reference for the incoming seminar of OM society on Riemannian geometry.
At the time this note is published to the course group, it is barely a draft, full of typos and fake proofs.
Also we don't have a preface or a bibliography.
I added this preface and bibliography to the note, and will print the final version of this note.
I tried to list the reference of every theorem, but there still are some I can't find the reference.
If more errors are found after the final reviewing, I will post an errata on GitHub repository.
The url for the repository is \url{https://github.com/matthewzenm/Riemannian-Geometry-notes}.
If you find anything that needs to be revised, please contact me at \url{matthewzenm@icloud.com}.

\begin{flushright}
    Zeng Mengchen

    2 March 2025
\end{flushright}