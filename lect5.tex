\chapter{Comparison Theorems}

\section{Rauch Comparison Theorem}

We first state and prove the Rauch comparison theorem.

\begin{thm}[Rauch comparison]\label{Rauch}
    Let $M^n,\tilde{M}^n$ be Riemannian manifolds, $\gamma:[0,a]\to M$, $\tilde{\gamma}:[0,a]\to\tilde{M}$ be unit speed geodesics, and $J,\tilde{J}$ be Jacobi fields along $\gamma,\tilde{\gamma}$ respectively, such that $J(0)=\tilde{J}(0)=0$, $\langle\dot{J}(0),\dot{\gamma}(0)\rangle=\left\langle\dot{\tilde{J}}(0),\dot{\tilde{\gamma}}(0)\right\rangle$, $|\dot{J}(0)|=\left|\dot{\tilde{J}}(0)\right|$.
    Assume that
    \begin{enumerate}[(1)]
        \item $\gamma$ has no conjugate points of $\gamma(0)$ along $\gamma$;
        \item $K_\gamma(\dot{\gamma},v)\geq K_{\tilde{\gamma}}\left(\dot{\tilde{\gamma}},\tilde{v}\right)$ with $|v|=|\tilde{v}|=1$.
    \end{enumerate}
    Then $|J(t)|\leq|\tilde{J}(t)|$ for all $t\in[0,a]$.
\end{thm}

\begin{rem}
    Before giving the proof of the theorem, we notice that when $\dim M=\dim\tilde{M}=2$, the theorem reduces to the Liouville--Sturm comparison theorem in ODE theory.
\end{rem}

\begin{proof}[Proof of Theorem~\ref{Rauch}]
    Decompose $J,\tilde{J}$ into $J^\perp(t)+(at+b)\dot{\gamma}(t)$, $\tilde{J}^\perp(t)+(\tilde{a}t+\tilde{b})\dot{\tilde{\gamma}}(t)$ respectively.
    Since $J(0)=\tilde{J}(0)=0$, we have $b=\tilde{b}=0$;
    Moreover, since $\langle\dot{J}(0),\dot{\gamma}(0)\rangle=\left\langle\dot{\tilde{J}}(0),\dot{\tilde{\gamma}}(0)\right\rangle$, we have $a=\tilde{a}$.
    Since $|J^\perp|^2=|J|^2-(at)^2$, $|\tilde{J}^\perp|^2=|\tilde{J}|^2-(\tilde{a}t)^2$,
    we can just compare $J^\perp$ and $\tilde{J}^\perp$.
    So we assume without loss of generality that $J,\tilde{J}$ are normal Jacobi fields, that is, $\dot{J}(0)\perp\dot{\gamma}(0)$, $\dot{\tilde{J}}(0)\perp\dot{\tilde{\gamma}}(0)$.

    Let $f(t)=|J(t)|^2$, $\tilde{f}(t)=|\tilde{J}(t)|^2$.
    By l'Hospital's rule, we have
    \[\lim_{\varepsilon\to 0}\frac{\tilde{f}(\varepsilon)}{f(\varepsilon)}=1,\]
    hence only need to show $\tilde{f}/f$ is monotonically increasing, which is equivalent to
    \[\frac{\dot{\tilde{f}}}{\tilde{f}}\geq\frac{\dot{f}}{f}\iff\frac{\left\langle\dot{\tilde{J}},\tilde{J}\right\rangle}{\left|\tilde{J}\right|^2}\geq\frac{\langle\dot{J},J\rangle}{|J|^2}.\]
    We can check the last inequality pointwisely, so we can scale the Jacobi fields to make $|\tilde{J}|=|J|$.
    Then we write the inequality into a lemma.
\end{proof}

\begin{lem}
    If $J,\tilde{J}$ are Jacobi fields along $\gamma,\tilde{\gamma}$ respectively, with $J(0)=\tilde{J}(0)=0$, $\dot{J}(0)\perp\dot{\gamma}(0)$, $\dot{\tilde{J}}(0)\perp\dot{\tilde{\gamma}}(0)$, and $|J(c)|=|\tilde{J}(c)|$.
    Moreover, the conditions (1) and (2) in Theorem~\ref{Rauch}~hold, then $\left\langle\dot{\tilde{J}}(c),\tilde{J}(c)\right\rangle\geq\langle\dot{J}(c),J(c)\rangle$.
\end{lem}
\begin{proof}
    First we notice that
    \[\langle\dot{J}(c),J(c)\rangle=I_{[0,c]}(J,J),\ \left\langle\dot{\tilde{J}}(c),\tilde{J}(c)\right\rangle=I_{[0,c]}\left(\tilde{J},\tilde{J}\right).\]
    From now on we will simply use $I$ as $I_{[0,c]}$.
    (Here we abuse the notation, using same $I$ to denote index form on different curves.
    We will write the curve in subscript in case of ambiguous if necessary.)
    Let $\{e_1(t),\cdots,e_{n-1}(t),\dot{\gamma}(t)\}$, $\left\{\tilde{e}_1(t),\cdots,\tilde{e}_{n-1}(t),\dot{\tilde{\gamma}}(t)\right\}$ be parallel orthonormal frames along $\gamma,\tilde{\gamma}$ respectively, such that $J(c)=\alpha e_1(c)$, $\tilde{J}(c)=\alpha\tilde{e}_1(c)$.
    Let
    \[J(t)=\sum_{i=1}^{n-1}h_i(t)e_i(t),\ \tilde{J}(t)=\sum_{i=1}^{n-1}\tilde{h}_i(t)\tilde{e}_i(t),\]
    then $h_i(c)=\tilde{h}_i(c)=0$, $i=2,\cdots,n-1$.
    Define $U(t)=\sum_{i=1}^{n-1}\tilde{h}_i(t)e_i(t)$ along $\gamma$, then $U(c)=J(c)$, $U(0)=J(0)=0$, $|U(t)|=|\tilde{J}(T)|$.
    Thus by Index Lemma, $I_\gamma(J,J)\leq I_\gamma(U,U)$.
    However, we have
    \begin{align*}
        I_\gamma(U,U)&=\int_0^c(|\dot{U}|^2-\langle R(\dot{\gamma},U)\dot{\gamma},U\rangle)\d{t}\\
        &=\int_0^c\left(\left|\dot{\tilde{J}}\right|^2-|U|^2K(\dot{\gamma},U/|U|)\right)\d{t}\\
        &\leq\int_0^c\left(\left|\dot{\tilde{J}}\right|^2-\left|\tilde{J}\right|K\left(\dot{\tilde{\gamma}},\tilde{J}/|\tilde{J}|\right)\right)\d{t}\\
        &=I_{\tilde{\gamma}}(\tilde{J},\tilde{J})
    \end{align*}
    Thus we have $\left\langle\dot{\tilde{J}}(c),\tilde{J}(c)\right\rangle\geq\langle\dot{J}(c),J(c)\rangle$.
\end{proof}

\begin{cor}
    If Riemannian manifold $(M^n,g)$ satisfies $\operatorname{Sect}\leq 0$, then $|J(t)|\geq t$, that is, $\left|\exp_{p*}|_v(w)\right|\geq|w|$.
\end{cor}
\begin{proof}
    Compare $(M^n,g)$ with Euclidean space $(\mathbb{R}^n,\delta)$.
\end{proof}

The corollary implies the following important theorem.
\begin{thm}\label{C-H 1}
    Let $M$ be a complete Riemannian manifold with $\operatorname{Sect}\leq 0$, $p\in M$ be any point.
    Then $M$ has no conjugate points of $p$.
\end{thm}

Theorem~\ref{C-H 1}~is known as the first part of the celebrated Cartan--Hadamard Theorem.
We state the rest part of the theorem as follows.
\begin{thm}[Cartan--Hadamard]
    Let $M^n$ be a complete Riemannian manifold with $\operatorname{Sect}\leq 0$, then the universal cover of $M$ is diffeomorphic to $\mathbb{R}^n$.
\end{thm}
\begin{proof}[Part of the Proof]
    Without loss of generality, we assume $M$ is simply connected.
    Then we only need to show $M$ is diffeomorphic to $\mathbb{R}^n$.
    Since $M$ is complete, $\exp_p:T_pM\to M$ is surjective, and by Theorem~\ref{C-H 1}, the $\exp_p$ is nondegenerate.
    Thus $\exp_p$ is a local isometry between $(T_pM,\exp_p^*g)\to(M,g)$.
    We show that $\exp_p^*g$ is complete.
    Let $\gamma_v(t):[0,+\infty)\to T_pM, t\mapsto vt$ for any unit vector $v\in T_pM$, then $\exp_p(\gamma)$ is a geodesic in $(M,g)$.
    By the definition of $\exp_p^*g$, $\gamma_v(t)$ is a geodesic in $(T_pM,\exp_p^*g)$, hence the exponential map $\exp_0:T_0(T_pM)\to T_pM$ is well-defined at $0\in T_pM$.
    Hence by Hopf--Rinow Theorem, $\exp_p^*g$ is complete.
    Now the proof reduces to the following proposition.
\end{proof}

\begin{prop}
    Let $f:M\to N$ be local isometry between Riemannian manifolds, with $M$ complete.
    Then $N$ is complete and $f$ is a covering map.
\end{prop}

We stop here, and refer to Peter Peterson's \emph{Riemannian Geometry}, 3rd ed., Lemma 5.6.4.

We mention something more.
Cartan--Hadamard Theorem can be used to prove the uniqueness of space forms, that is, simply connected complete constant sectional curvature Riemannian manifolds are unique up to an isometry.
This can be found in Wu Hung--Hsi et.\ al.'s \emph{Introduction to Riemannian Geometry}, Chapter 5.

\section{Covariant Derivative of Tensors}
We interrupt our discussion on comparison theorems and turn to covariant derivative of tensors.
This is a relatively independent discussion from our main theme, but will be used later discussion.
For simplicity, we consider only covariant tensors.

\begin{defn}
    Let $T$ be a $(0,r)$-tensor ($r\geq 0$), then we define the \textbf{covariant derivative} of $T$ as
    \[\nabla_XT(Y_1,\cdots,Y_r)=XT(Y_1,\cdots,Y_r)-\sum_{i=1}^rT(Y_1,\cdots,\nabla_XY_i,\cdots,Y_r)\]
    for any vector fields $Y_1,\cdots,Y_r$.
    Notice that if $T$ is a function (i.e., $(0,0)$-tensor), the minus summation term does not exist.
    Moreover, we define the \textbf{covariant differential} of $T$ as
    \[\nabla T(Y_1,\cdots,Y_r,X)=\nabla_XT(Y_1,\cdots,Y_r).\]
\end{defn}

One can check $\nabla T$ is indeed tensorial, and satisfies Leibniz rule, that is
\[\nabla(T\otimes S)=(\nabla T)\otimes S+T\otimes(\nabla S).\]

We look at covariant derivative locally.
It's known that $(0,r)$-tensors have a basis
\[\{\d{x^{i_1}}\otimes\cdots\otimes\d{x^{i_r}}\}_{1\leq i_1,\cdots,i_r\leq n},\]
(We will not prove here, please refer to any book on differentiable manifolds)
so we let
\[T=T_{i_1\cdots i_r}\d{x^{i_1}}\otimes\cdots\otimes\d{x^{i_r}},\]
and write
\[\nabla_{\partial_k}T=T_{i_1\cdots i_r;k}\d{x^{i_1}}\otimes\cdots\otimes\d{x^{i_r}}.\]
Let's compute the coefficient. First we compute
\[\nabla_{\partial_i}\d{x^j}=a_k\d{x^k},\]
then
\begin{align*}
    a_k&=\nabla_{\partial_i}\d{x^j}(\partial_k)\\
    &=-\d{x^j}(\nabla_{\partial_i}\partial_k)\\
    &=-\d{x^j}(\Gamma^l_{ik}\partial_l)\\
    &=-\Gamma^j_{ik},
\end{align*}
that is,
\[\nabla_{\partial_i}d{x^j}=-\Gamma^j_{ik}d{x^k}.\]
Using this, we can compute
\begin{align*}
    &\nabla_{\partial_k}(T_{i_1\cdots i_r}\d{x^{i_1}}\otimes\cdots\otimes\d{x^{i_r}})\\
    =&\partial_kT_{i_1\cdots i_r}\d{x^{i_1}}\otimes\cdots\otimes\d{x^{i_r}}+T_{i_1\cdots i_r}\sum_{l=1}^r\d{x^{i_1}}\otimes\cdots\otimes(\nabla_{\partial_k}\d{x^{i_j}})\otimes\cdots\otimes\d{x^{i_r}}\\
    =&\partial_kT_{i_1\cdots i_r}\d{x^{i_1}}\otimes\cdots\otimes\d{x^{i_r}}+T_{i_1\cdots i_r}\sum_{l=1}^r\d{x^{i_1}}\otimes\cdots\otimes(-\Gamma^{i_j}_{kp}\d{x^p})\otimes\cdots\otimes\d{x^{i_r}}\\
    =&\partial_kT_{i_1\cdots i_r}\d{x^{i_1}}\otimes\cdots\otimes\d{x^{i_r}}-\sum_{l=1}^rT_{i_1\cdots i_{l-1}pi_{l+1}\cdots i_r}\Gamma^p_{ki_l}\d{x^{i_1}}\otimes\cdots\otimes\d{x^{i_r}},
\end{align*}
hence we have
\[T_{i_1\cdots i_r;k}=\partial_kT_{i_1\cdots i_r}-\sum_{l=1}^rT_{i_1\cdots i_{l-1}pi_{l+1}\cdots i_r}\Gamma^p_{ki_l}.\]

Moreover, we can consider second covariant derivative.
\begin{symb}
    We use the symbol $\nabla^2_{X,Y}T$ to denote the tensor 
    \[\nabla^2_{X,Y}T(Y_1,\cdots,Y_r)=\nabla(\nabla T)(Y_1,\cdots,Y_r,Y,X).\]
\end{symb}

\begin{prop}
    We have
    \begin{equation}
        \nabla^2_{X,Y}T=\nabla_X(\nabla_YT)-\nabla_{\nabla_XY}T.\label{2nd cov}
    \end{equation}
\end{prop}
\begin{proof}
    We have
    \begin{align*}
        &\nabla^2_{X,Y}T(Y_1,\cdots,Y_r)\\
        =&\nabla(\nabla T)(Y_1,\cdots,Y_r,Y,X)\\
        =&\nabla_X(\nabla T)(Y_1,\cdots,Y_r,Y)\\
        =&X(\nabla T)(Y_1,\cdots,Y_r,Y)-\sum_{i=1}^r(\nabla T)(Y_1,\cdots,\nabla_XY_i,\cdots,Y_r,Y)\\
        &-(\nabla T)(Y_1,\cdots,Y_r,\nabla_XY)\\
        =&X(\nabla_YT)(Y_1,\cdots,Y_r)-\sum_{i=1}^r(\nabla_YT)(Y_1,\cdots,\nabla_XY_i,\cdots,Y_r)\\
        &-(\nabla_{\nabla_XY}T)(Y_1,\cdots,Y_r)\\
        =&\nabla_X(\nabla_YT)(Y_1,\cdots,Y_r)-(\nabla_{\nabla_XY}T)(Y_1,\cdots,Y_r).\qedhere
    \end{align*}
\end{proof}

Next we discuss curvature of tensors.

\begin{defn}[Curvature operator]
    Let $X,Y$ be vector fields, define an endomorphism $R(X,Y)$ of $(0,r)$-tensors as
    \[R(X,Y)T=\nabla_Y\nabla_XT-\nabla_X\nabla_YT+\nabla_{[X,Y]}T.\]
\end{defn}

\begin{prop}[Ricci identity]
    For a $(0,r)$-tensor $T$ and vector fields $X,Y$, we have
    \[(\nabla^2_{Y,X}-\nabla^2_{X,Y})T=R(X,Y)T\]
\end{prop}
\begin{proof}
    Using equation~\ref{2nd cov}, we have
    \begin{align*}
        (\nabla^2_{Y,X}-\nabla^2_{X,Y})T&=\nabla_Y\nabla_XT-\nabla_{\nabla_YX}T-\nabla_X\nabla_YT+\nabla_{\nabla_XY}T\\
        &=\nabla_Y\nabla_XT-\nabla_X\nabla_YT+\nabla_{(\nabla_XY-\nabla_YX)}T\\
        &=\nabla_Y\nabla_XT-\nabla_X\nabla_YT+\nabla_{[X,Y]}T\quad\text{(torsion-freeness)}\\
        &=R(X,Y)T.\qedhere
    \end{align*}
\end{proof}

Now we introduce some differential operators.

\begin{defn}
    If $S$ is a symmetric $(0,2)$-tensor on  a Riemannian manifold $(M,g)$, we define the \textbf{trace} of $S$ to be
    \[\tr_g(S)|_p=\sum_{i=1}^nS_p(e_i,e_i),\ \forall p\in M\]
    for any orthonormal basis $\{e_i\}$ of $T_pM$.
    It can be checked that the definition does not depend on the choice of orthonormal basis.
    By this definition, trace is a smooth function on $M$.
\end{defn}

\begin{defn}
    Let $(M,g)$ be a Riemannian manifold, $f\in C^\infty(M)$.
    Define the \textbf{Hessian} of $f$ to be a $(0,2)$-tensor $\nabla^2f$.
    Ricci identity shows $\nabla^2f$ is symmetric, hence we define its trace to be the \textbf{Laplacian} of $f$, denoted by $\Delta f$.
\end{defn}

\begin{prop}
    Let $(M,g)$ be a Riemannian manifold, $f\in C^\infty(M)$.
    Then we have
    \[\nabla^2f(X,Y)=Y(Xf)-(\nabla_YX)(f).\]
\end{prop}
\begin{proof}
    This is just the equation~\ref{2nd cov}.
\end{proof}

We introduce the gradient of a function.
\begin{defn}
    Let $(M,g)$ be a Riemannian manifold, $f\in C^\infty(M)$.
    Then we define the \textbf{gradient} of $f$ to be the vector field $\nabla f$ defined by $g(\nabla f,X)=Xf$ for any $X\in\mathfrak{X}(M)$.
\end{defn}

We now give the local expression of $\nabla f$ (and show the existence of $\nabla f$ as a by-product).
Let $\nabla f=f^i\partial_i$, then we have
\[g(f,\partial_j)=f^ig_{ij}=\partial_jf.\]
Then use $g^{ij}$ to denote the inverse matrix of $g_{ij}$, we have
\[f^i=g^{ij}\partial_jf,\]
that is,
\[\nabla f=g^{ij}\frac{\partial f}{\partial x^{j}}\frac{\partial{}}{\partial{x^i}}.\]

Finally we define the norm of a symmetric $(0,2)$-tensor.
This will be used in Bochner formula in next lecture.

\begin{defn}
    Let $S$ be a symmetric $(0,2)$-tensor.
    At $p\in M$, let $\{e_i\}$ be an orthonormal basis, $\{f^i\}$ be its dual basis.
    Let $S=S_{ij}f^i\otimes f^j$, then define the norm of $S$ to be
    \[|S|^2:=\sum_{i,j}|S_{ij}|^2.\]
\end{defn}

We also write the norm of $S$ in general coordinates.
One can check under general $g$ we have
\[|S|^2=g^{ij}g^{kl}S_{ij}S_{kl},\]
for $S=S_{ij}\d{x^i}\otimes\d{x^j}$.

\section{Distance Function and Cut Points}
