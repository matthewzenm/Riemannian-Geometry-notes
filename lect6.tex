\chapter{Bochner Formula and Application}
We state and prove the useful Bochner formula, and apply it on distance function.
We will give another proof of Laplace comparison theorem using Bochner formula, and prove Cheeger--Gromoll splitting theorem.

\section{Bochner Formula}

\begin{thm}[Bochner formula]
    Let $M$ be a Riemannian manifold, $f\in C^\infty(M)$. Then we have
    \[\Delta\frac{1}{2}|\nabla f|^2=|\nabla^2 f|^2+\langle\nabla\Delta f,\nabla f\rangle+\langle\Ric(\nabla f),\nabla f\rangle.\]
\end{thm}

We first need a lemma.
\begin{lem}
    Under geodesic normal coordinate around $p$, the Ricci identity (Proposition~\ref{Ricci identity}) of $1$-form is equivalent to
    \[f_{;kij}-f_{;kji}=-f_{;l}R^l_{ijk},\]
    where $f_{;i}$ means $\partial_if$.
\end{lem}
\begin{proof}
    First we notice that
    \begin{align*}
        \nabla_{\partial_i}\nabla_{\partial_j}\d{f}&=\nabla_{\partial_i}(\partial_jf_{;k}\d{x^k}+f_{;k}\d{x^k})\\
        &=\partial_i\partial_jf_{;k}\d{x^k}-\nabla_{\partial_i}\Gamma_{jl}^k\d{x^l}\\
        &=f_{;kji}\d{x^k}
    \end{align*}
    since all $\Gamma^k_{jl}=0$ at $p$.
    Then we have
    \begin{align*}
        (R(\partial_i,\partial_j)\d{f})(\partial_k)&=(\nabla_{\partial_j}\nabla_{\partial_i}-\nabla_{\partial_i}\nabla_{\partial_j}+\nabla_{[\partial_i,\partial_j]})\d{f}(\partial_k)\\
        &=(f_{;kij}-f_{;kji}).
    \end{align*}
    On the other hand, we have
    \begin{align*}
        (R(\partial_i,\partial_j)\d{f})(\partial_k)&=-\d{f}(R(\partial_i,\partial_j)\partial_k)\\
        &=-\d{f}(R_{ijk}^l\partial_l)\\
        &=-f_{;m}\d{x^m}(R_{ijk}^l\partial_l)\\
        &=-f_{;l}R_{ijk}^l,
    \end{align*}
    hence the equality holds.
\end{proof}

\begin{proof}[Proof of Bochner formula]
    Under geodesic normal coordinate we have the following calculation
    \begin{align*}
        \Delta\frac{1}{2}|\nabla f|^2&=\sum_{i,j}\left(\frac{1}{2}f^2_{;j}\right)_{;ii}\\
        &=\sum_{i,j}(f_{;j}f_{;ji})_{;i}\\
        &=\sum_{i,j}(f^2_{;ji}+f_{;j}f_{;jii})\\
        &=\sum_{i,j}(f^2_{;ij}+f_{;j}f_{;iji})\quad\text{(Hessian is interchangeable)}\\
        &=\sum_{i,j}f^2_{;ij}+\sum_{i,j}f_{;j}(f_{;iij}-f_{;k}R_{jii}^k)\quad\text{(Ricci identity)}\\
        &=\sum_{i,j}f^2_{;ij}+\sum_{i,j}f_{;j}f_{;iij}+\sum_{i,j}f_{;k}R_{iji}^k\\
        &=\sum_{i,j}f^2_{;ij}+\sum_{i,j}f_{;iij}f_{;j}+\sum_{j}f_{;j}f_{;k}\Ric_j^k\\
        &=|\nabla f|^2+\langle\Delta\nabla f,\nabla f\rangle+\langle\Ric(\nabla f),\nabla f\rangle.\qedhere
    \end{align*}
\end{proof}

As a first application, we use Bochner formula to provide an alternative proof of Bishop's theorem (Theorem~\ref{Bishop}).
\begin{proof}[Alternative proof of Theorem~\ref{Bishop}]
    We apply Bochner formula on distance function $d_p$.
    Notice that $|\nabla d_p|=1$, thus we have
    \[0=|\nabla^2d_p|^2+\langle\nabla\Delta d_p,\nabla d_p\rangle+\langle\Ric(\nabla d_p),\nabla d_p\rangle.\]
    Now we use the geodesic polar coordinate.
    First let $L$ by the corresponding linear transformation of $\nabla^2d_p$, by Proposition~\ref{hess dp in tangential}~and Proposition~\ref{hess dp in other}, we know that $L$ has $n-1$ nonzero eigenvalue, say $\lambda_1,\cdots,\lambda_{n-1}$.
    Then by Cauchy--Schwartz inequality, we have
    \begin{align*}
        |\nabla^2d_p|^2&=\tr(L\circ L)=\sum_{k=1}^{n-1}\lambda^2_k\\
        &\geq\frac{1}{n-1}\left(\sum_{k=1}^{n-1}\lambda_k\right)^2=\frac{1}{n-1}(\tr L)^2\\
        &=\frac{1}{n-1}(\Delta d_p)^2.
    \end{align*}
    Next, we notice that $\nabla d_p=\partial_r$, hence
    \[\langle\nabla\Delta d_p,\nabla d_p\rangle=\langle\partial_r(\Delta d_p)\partial_r,\partial_r\rangle=\frac{\partial(\Delta d_p)}{\partial{r}}.\]
    Moreover, by the assumption, we have
    \[\langle\Ric(\nabla d_p),\nabla d_p\rangle\geq(n-1)K\langle\nabla d_p,\nabla d_p\rangle=(n-1)K.\]
    Add three equalities together, and since $\Delta d_p=\partial_r(\log\mathscr{J})$, we have
    \begin{equation}
        \frac{\partial^2{}}{\partial{r^2}}(\log\mathscr{J})+\frac{1}{n-1}\left(\frac{\partial{}}{\partial{r}}\log\mathscr{J}\right)+(n-1)K\leq 0.\label{eq in bishop}
    \end{equation}
    Let $\Phi=\mathscr{J}^{1/(n-1)}$, then the equation~\eqref{eq in bishop}~is equivalent to
    \[\ddot{\Phi}+K\Phi\leq 0.\]
    However, let $\Phi_K=\mathscr{J}_K^{1/(n-1)}$, we have $\ddot{\Phi}_K+K\Phi_K=0$, and $\Phi_K(0)=\Phi(0)=0, \Phi'_K(0)=\Phi'(0)=1$.
    Then by ODE theory, we have $\Phi/\Phi_K$ is nonincreasing.
    This derives $\mathscr{J}/\mathscr{J}_K$ is nonincreasing.
\end{proof}