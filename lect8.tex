\chapter{A Little Thing on Bernstein Problem}
In our last topic class we discuss the Bernstein problem for minimal surfaces.
We first introduce some motivation for the problem.

Let $u:\Omega\subset\mathbb{R}^{n-1}\to\mathbb{R}$ be a function, then its graph $(x^1,\cdots,x^{n-1},u(x^1,\cdots,x^{n-1}))$ is a minimal surface if and only if it satisfies the following \emph{minimal surface equation} (MSE):\index{minimal surface equation}
\[\div\left(\frac{\nabla u}{\sqrt{1+|\nabla u|^2}}\right)=0.\]
When $n=3$, MSE is found by Lagrange in 1762.

If we do not restrict us on graph, then we have the following Plateau problem.
\begin{pro}[Plateau]\index{Plateau problem}
    Let $\Gamma\subset\mathbb{R}^3$ be a simple closed curve, find an area-minimizing surface $\Sigma\subset\mathbb{R}^3$ such that $\partial\Sigma=\Gamma$.
\end{pro}

This was solved by Douglas--Rado in 1930s.
Please refer to Struwe's \emph{Calculus of Variation} for more discussion.

But for minimal graph (i.e.\ a minimal surface that is a graph)\index{minimal graph}, we have the following Bernstein's theorem.
\begin{thm}[Bernstein, 1916]\index{theorem!Bernstein}
    Let $u:\mathbb{R}^2\to\mathbb{R}$ be a function that satisfies MSE (called \emph{entire graph}), then $u$ is an affine function, that is
    \[u(x)=\vec{a}\cdot x+b.\]
\end{thm}

Then we have the Bernstein problem.
\begin{pro}[Bernstein]\index{Bernstein problem}
    Let $u:\mathbb{R}^{n-1}\to\mathbb{R}$ be an entire graph in $\mathbb{R}^n$.
    Is the graph of $u$ an affine graph?
    In other words, is entire minimal graph in $\mathbb{R}^n$ a hyperplane?
\end{pro}

Bernstein problem is also solved.
The timeline is
\begin{itemize}
    \item Bernstein solved $n=3$ case in 1916.
    \item Fleming gave an important alternative method in 1960s.
    \item De Giorgi solved $n=4$ case enlightened by Fleming.
    \item Almgren solved $n=5$ case in 1966.
    \item J.\ Simons solved $n\leq 8$ cases in 1968.
    \item The problem is false when $n\geq 9$.
    Counterexample was given by Bombien--de~Giorgi--Ginsti in 1969.
\end{itemize}

We have the following fact.
\begin{prop}\index{minimal graph!area-minimizing}
    A minimal graph in $\mathbb{R}^n$ is area-minimizing.
\end{prop}
\begin{proof}
    We need to show if $\Sigma_u$ is a minimal graph, $\Sigma$ is a surface such that $\partial\Sigma=\partial\Sigma_u$, then $|\Sigma_u|\leq|\Sigma|$.
    Define a vector field $X\in\mathfrak{X}(\mathbb{R}^n)$, such that
    \[X(x^1,\cdots,x^n)=\frac{(-\nabla u,1)}{\sqrt{1+|\nabla u|^2}}.\]
    Since $X$ has nothing to do with $x^n$, we have
    \[\div{X}=\div\nolimits_{\mathbb{R}^{n-1}}\left(\frac{\nabla u}{\sqrt{1+|\nabla u|^2}}\right)=0.\]
    Let $\Omega$ be the domain enclosed by $\Sigma_u$ and $\Sigma$, $\nu_\Sigma$ and $\nu_{\Sigma_u}$ be the outward unit normal vector field of $\Sigma$ and $\Sigma_u$.
    Then by divergence theorem, we have
    \begin{align*}
        0&=\int_\Omega\div\nolimits_{\mathbb{R}^n}X\\
        &=\int_\Sigma\langle X,\nu_\Sigma\rangle+\int_{\Sigma_u}\langle X,\nu_{\Sigma_u}\rangle.
    \end{align*}
    Since $\nu_{\Sigma_u}=\frac{(\nabla u,-1)}{\sqrt{1+|\nabla u|^2}}$, we have $\langle X,\nu_{\Sigma_u}\rangle=-1$.
    Thus by Cauchy--Schwarz inequality, we have
    \[|\Sigma_u|\leq\int_\Sigma\langle X,\nu_\Sigma\rangle\leq\int_\Sigma|X||\nu_\Sigma|=|\Sigma|.\qedhere\]
\end{proof}

\begin{rem}
    This fact is closely related to calibrated geometry.
    We refer to Harvey and Lawson's \emph{Calibrated Geometries} (1982).
\end{rem}

\begin{cor}
    Let $\Sigma_u$ be an entire minimal graph, and $B_R\subset\mathbb{R}^n$.
    Then
    \[A(\Sigma_n\cap B_r)\leq\frac{1}{2}\omega_{n-1}R^{n-1},\]
    that is, $\Sigma_u$ has Euclidean area growth.
\end{cor}
\begin{proof}
    $\Sigma\cap B_R$ separates $B_R$ into upper part $B^1_R$ and lower part $B^2_R$.
    Then we have
    \[|\Sigma_u|\leq\min\{|\partial B^1_R|,|\partial B^2_R|\}\leq\frac{1}{2}\omega_{n-1}R^{n-1}.\qedhere\]
\end{proof}

We now discuss the conception of stability.
This is used to solve the problem that when minimal surfaces actually minimizes area.
\begin{defn}[Stability]\index{stability}
    We call the surface $\Sigma$ is \emph{stable} if for any proper variation $f:\Sigma\times(-\varepsilon,\varepsilon)\to M$ the second variation is nonnegative, that is,
    \[\left.\frac{\d^2{}}{\d{t^2}}\right|_{t=0}|\Sigma_t|=\int_\Sigma(|\nabla^\perp X|^2-|\langle\II,X\rangle|^2-\tr_\Sigma\langle R^M(\bullet,X)\bullet,X\rangle)\geq 0,\]
    where $X$ is the variation vector field.
    When $\Sigma$ has codimension $1$, let $X=\varphi N$ with $N$ be a unit normal vector field, $\varphi\in C^\infty_c(\Sigma)$.
    Then the condition is equivalent to
    \[\int_\Sigma(|\nabla\varphi|^2-|h|^2\varphi^2-\Ric(N,N)\varphi^2)\geq 0\]
    or
    \[\int_\Sigma-\varphi(\Delta+|h|^2+\Ric(N,N))\varphi\geq 0,\]
    where $(\Delta+|h|^2+\Ric(N,N))$ is the \emph{Jacobi operator}.
\end{defn}

About stability, we have the following theorem.
\begin{thm}[Fischer--Colbrie--Schoen]\index{theorem!Fischer--Colbrie--Schoen}
    A surface $\Sigma$ is stable if and only if there is a $u\in C^\infty(\Sigma)$ with $u>0$, such that
    \[(\Delta+|h|^2+\Ric(N,N))u=0.\]
\end{thm}

We now give a proof of $2$-dimensional Bernstein problem.
This is a corollary of following theorem.
\begin{thm}\label{prebernstein}
    Let $u:\Omega\subset\mathbb{R}^2\to\mathbb{R}$ be a minimal graph, $B^2_R\subset\Omega$.
    Then
    \[\int_{B^3_{\sqrt{R}}\cap\Sigma_u}|h|^2\leq\frac{C}{\log{R}}.\]
\end{thm}

\begin{proof}
    Since minimal graph minimizes area, the second variation of area must be positive, and thus minimal graph is stable.
    Therefore for any $\varphi\in C^\infty_c(\Sigma)$
    \[\int_\Sigma|h|^2\varphi^2\leq\int_\Sigma|\nabla_\Sigma\varphi|^2.\]
    Choose
    \[\varphi(x)=\begin{cases}
        1, & |x|\leq\sqrt{R},\\
        2\left(1-\frac{\log|x|}{\log{R}}\right), & \sqrt{R}\leq|X|\leq R,\\
        0, & |x|\geq R,
    \end{cases}\]
    then we have an estimate
    \[|\nabla_\Sigma\varphi|\leq|\nabla_{\mathbb{R}^3}\varphi|\leq\frac{1}{\log{R}}\cdot\frac{1}{|x|}.\]
    Therefore we have
    \[\int_{\Sigma\cap B^3_R}|\nabla_\Sigma\varphi|^2\leq\sum_{k=\log{R}/2}^{\log{R}}\int_{\Sigma\cap(B^3_{e^k}\backslash B^3_{e^{k-1}})}\frac{1}{(\log{R})^2}\cdot\frac{1}{|x|^2}.\]
    (We are mainly estimating the order of $h$, so we can adjust $R$ to make $\log{R}$ an integer.)
    But we have
    \[|\Sigma\cap B^3_{e^k}|\leq ce^{2k},\ \frac{1}{(\log{R})^2}\cdot\frac{1}{|x|^2}\leq e^{-2(k-1)},\]
    hence
    \begin{align*}
        \int_{\Sigma\cap B^3_R}|\nabla_\Sigma\varphi|^2&\leq ce^2\log{R}\cdot\frac{1}{(\log{R})^2}\\
        &=\frac{C}{\log{R}}.\qedhere
    \end{align*}
\end{proof}
\begin{rem}
    This argument is called logarithm cutoff argument.
\end{rem}

Before we procced our discuss, we need the following monotonicity formula.
\begin{thm}[Monotonicity formula]\index{monotonicity formula}
    Let $\Sigma^k\subset\mathbb{R}^n$ be a minimal submanifold, $x_0\in\mathbb{R}^n$.
    Then for $r>s$ we have
    \[\frac{|\Sigma^k\cap B^n_r(x_0)|}{r^k}-\frac{|\Sigma^k\cap B^n_s(x_0)|}{r^k}=\int_{(B^n_r(x_0)\backslash B^n_s(x_0))\cap\Sigma}\frac{|(x-x_0)^\perp|^2}{|x-x_0|^{k+2}}\d{A}.\]
\end{thm}
\begin{proof}
    Without loss of generality we assume $x_0=0$.
    We have
    \[\frac{\d{}}{\d{r}}\left(\frac{|\Sigma^k\cap B_r|}{r^k}\right)=r^{-k}\int_{\Sigma\cap\partial B_r}\frac{1}{|\nabla_\Sigma|x||}-kr^{-k-1}|\Sigma\cap B_r|,\]
    here we use the coarea formula
    \[|\Sigma^k\cap B_r|=|\{x\in\Sigma|\ |x|\leq r\}=\int_0^r\int_{\Sigma\cap\partial B_r}\frac{1}{|\nabla_\Sigma|x||},\]
    a proof can be found on Mei Jiaqiang's \emph{Mathematical Analysis}.
    Since $\Sigma^k$ is minimal, we have $\div\nolimits_{\Sigma\cap B_r}(x^\perp)=0$ as in Minkowski's formula.
    Then
    \begin{align*}
        k|\Sigma\cap B_r|&=\int_{\Sigma\cap B_r}\div\nolimits_{\Sigma}(x)\\
        &=\int_{\Sigma\cap B_r}\div\nolimits_{\Sigma}(x^\top)\\
        &=\int_{\Sigma\cap\partial B_r}\langle x^\top,\mu\rangle\quad\text{($\mu$ is the outward unit normal vector field)}\\
        &=\int_{\Sigma\cap\partial B_r}|x^\top|\quad\text{($\mu=x^\top/|x^\top|$)}\\
        &=r\int_{\Sigma\cap\partial B_r}\frac{|\nabla_\Sigma|x||^2}{|\nabla_\Sigma|x||}\quad\text{($|x^\top|=|\nabla_\Sigma|x||\cdot|x|$)}\\
        &=\frac{1}{r}\int_{\Sigma\cap\partial B_r}\frac{|x|^2-|x^\perp|^2}{|\nabla_\Sigma|x||}\\
        &=r\int_{\Sigma\cap\partial B_r}\frac{1}{|\nabla_\Sigma|x||}-\frac{1}{r}\int_{\Sigma\cap\partial B_r}\frac{|x^\perp|^2}{|\nabla_\Sigma|x||}.
    \end{align*}
    Hence
    \[\frac{\d{}}{\d{r}}\left(\frac{|\Sigma\cap B_r}{r^k}\right)=r^{-k-2}\int_{\Sigma\cap\partial B_r}\frac{|x^\perp|^2}{|\nabla_\Sigma|x||}.\qedhere\]
\end{proof}

Monotonicity formula is useful with the notion of cone.
We now introduce the conception of cone.

\begin{defn}\index{cone}
    Let $\Sigma^{k-1}\subset\mathbb{S}^{n-1}$ be a $(k-1)$-dimensional submanifold in $\mathbb{S}^{n-1}$.
    Then the \emph{cone} over $\Sigma$ is defined as
    \[\mathscr{C}_\Sigma:=\left\{x\in\mathbb{R}^n\left|\ \frac{x}{|x|}\in\Sigma\right.\right\}.\]
\end{defn}

There is a fact that
\begin{prop}
    $\Sigma\subset\mathbb{S}^{n-1}$ is minimal if and only if $\mathscr{C}_\Sigma$ is a minimal cone.
\end{prop}

One can observe that for a cone $\mathscr{C}$, $\frac{|\mathscr{C}\cap B_r|}{r^k}$ is constant.
Conversely, by monotonicity formula, for a minimal submanifold $\Sigma\subset\mathbb{R}^n$ the following are equivalent:
\begin{enumerate}[(1)]
    \item $\frac{|\Sigma\cap B_r|}{r^k}$ is constant;
    \item $x^\perp\equiv 0$;
    \item $\Sigma$ is a cone.
\end{enumerate}

Now we can outline the proof of Bernstein problem.
\begin{enumerate}
    \item Fleming:
    If every area-minimizing cone is flat, then the area-minimizing hypersurface in $\mathbb{R}^n$ is flat.
    This is proved by blow-down argument.
    Choose $p\in\Sigma$, let $\Sigma_r:=\frac{1}{r}(\Sigma\backslash\{p\})$.
    Let $r\to\infty$, $\Sigma_r$ converges to $\mathscr{C}$.
    By geometric measure theory, $\mathscr{C}$ is a set of finite perimeter.
    We need to show $\mathscr{C}$ is area-minimizing.
    We have
    \[\frac{|\mathscr{C}\cap B_r|}{R^{n-1}}=\lim_{r\to\infty}\frac{|\Sigma_r\cap B_R|}{R^{n-1}}=\lim_{r\to\infty}\frac{|\Sigma\cap B_{Rr}|}{(Rr)^{n-1}}=\theta_R=\text{const}.\]
    Thus $\mathscr{C}$ is a cone.
    By assumption, $\mathscr{C}$ is flat, then $\frac{|\mathscr{C}\cap B_R|}{R^{n-1}}=\omega_{n-1}$.
    Therefore
    \[\omega_{n-1}\leq\frac{|\Sigma\cap B_r|}{r^{n-1}}\leq\lim_{r\to\infty}\frac{\Sigma\cap B_r}{r^{n-1}}=\frac{|\mathscr{C}\cap B_r|}{r^{n-1}}=\omega_{n-1},\]
    hence $\Sigma$ is a cone, and therefore is flat.
    \item Simons 1968:
    Area-minimizing cone with possible singularity at $0$ in $\mathbb{R}^n$ ($n\leq 7$) is flat.
    Roughly speaking, the proof is analysing stability and the following Simons' inequality\footnote{We think Prof.\ Xia mistakenly wrote ``identity''.}
    \[\Delta_\Sigma|h|^2\geq -2|h|^4+2\left(1+\frac{2}{n-1}\right)|\nabla_\Sigma|h||^2.\]
    \item De Giorgi then solved $n=8$ case.\footnote{Prof.\ Xia provided a vague argument, so we omit it here.}
    \item In $\mathbb{R}^8$ we have the Simons cone $\{(x,y)\in\mathbb{R}^4\times\mathbb{R}^4|\ |x|=|y|\}$ being area-minimizing but not flat.
\end{enumerate}

Finally we discuss the stable Bernstein problem.
\begin{pro}[Stable Bernstein]\index{stable Bernstein problem}
    Let $\Sigma^{n-1}\subset\mathbb{R}^n$ ($n\leq 7$) be a stable complete two-sided minimal hypersurface.
    Is $\Sigma$ flat?
\end{pro}

We have the following progress.
\begin{itemize}
    \item $n=3$ case is solved by Fischer--Colbrie--Schoen--do~Carmo--Peng in 1980.
    \item $n=4$ case is solved by Chodosh--Li in 2021.
    Catino--Mastrolia--Roncoroni gave an alternative approach in 2023.
    \item $n=5$ case is solved by Chodosh--Li--Stryker in 2024/01.
    \item $n=6$ case is solved by Mazet in 2024/05.
    \item $n=7$ case is still open.
\end{itemize}

On the other hand, we have the following results.
\begin{itemize}
    \item Schoen--Simon--Yau 1976: Assume $|\Sigma\cap B_R|\leq CR^{n-1}|$, then stable Bernstein problem is true for $n\leq 6$.
    \item Bellettini 2023: the $n=7$ case of extensions of above theorem is true.
\end{itemize}
These two results uses de Giorgi--Mozer iteration on Simons inequality.

At last we give some reference books.
\begin{itemize}
    \item Colding--Minicozi: \emph{A Course in Minimal Surface}.
    \item L. Simon: \emph{Geometric Measure Theory}.
    \item F. Maggi: \emph{Geometric Measure Theory}.
    \item Y. Xin: \emph{Geometry of Submanifold}.
    \item Chodosh: Lecture notes.   
\end{itemize}