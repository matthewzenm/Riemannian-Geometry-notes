\chapter{Geodesics and Curvature}
In this lecture, we first introduce the concepts of geodesics and exponential maps.
By differentiating exponential map, we can introduce the concept of curvature and Jacobi fields.
Using exponential map, we can also introduce geodesic normal coordinate and geodesic polar coordinate.
As an application, we will use geodesic polar coordinate to show geodesics are locally length-minimizing.
Finally, we will introduce the notion of conjugate points.

\section{Geodesics and Exponential Maps}
\begin{defn}\index{geodesic}
    A curve $\gamma:[0,1]\to M$ is called a \emph{geodesic} if $\nabla_{\dot{\gamma}(t)}\dot{\gamma}(t)=0$.
\end{defn}

\begin{rem}
    Geodesics are constant speed, i.e., $|\dot{\gamma}(t)|\equiv\text{const}$.
    This can be shown by $\frac{\d{}}{\d{t}}\langle\dot{\gamma}(t),\dot{\gamma}(t)\rangle=2\langle\nabla_{\dot{\gamma}(t)}\dot{\gamma}(t),\dot{\gamma}(t)\rangle=0$.
\end{rem}

In a local chart $(U,\varphi)$, let $\varphi\circ\gamma(t)=(x^1(t),\cdots,x^n(t))$, then $\dot{\gamma}(t)=\dot{x}^i(t)\left.\frac{\partial{}}{\partial{x^i}}\right|_{\gamma(t)}$.
Thus the geodesic equation is
\[\nabla_{\dot{\gamma}(t)}\dot{\gamma}(t)=0\iff\ddot{x}^k(t)+\Gamma^k_{ij}(\gamma(t))\dot{x}^i(t)\dot{x}^j(t)=0,\ k=1,\cdots,n,\]
with $\gamma(0)=p,\dot{\gamma}(0)=v\in T_pM$.

Since the solution of an ODE relies continuously on initial value, we have the following proposition.
\begin{prop}\label{geodesic in nbhd}
    For any $p\in M$, there exists a neighborhood $V$ of $p$, such that there exists $\delta>0,\varepsilon>0$ and a differentiable map $\gamma:(-\delta,\delta)\times\mathscr{U}\to M$, where $\mathscr{U}=\{(q,v)\in TV|\ q\in V,\ v\in T_pM,\ |v|<\varepsilon\}$, such that $\gamma(t;q,v)$ is a geodesic with $\gamma(0)=q,\dot{\gamma}(0)=v$.
\end{prop}
The idea of the proof is the ODE theorem we mentioned above, but the proof is not simply using only ODE theory.
A proof of an equivalent proposition can be found in Wu Hung--Hsi, et.\ al.'s \emph{Introduction to Riemannian Geometry} Chapter 3, Lemma 1.

Observe that $\gamma(\lambda t;p,v)=\gamma(t;p,\lambda v)$.
Denote $\gamma(t;p,v)=\gamma_v(t)$, then above observation can be written as $\gamma_{\lambda v}(t)=\gamma_v(\lambda t)$.
Therefore, we can shorten the initial vector to lengthen the domain of geodesic.

\begin{defn}[Exponential map]\index{exponential map}
    Let $U\subset T_pM$ be a neighborhood of origin, such that for any $v\in U$, $\gamma_v(1)$ is defined (such neighborhood exists by Proposition~\ref{geodesic in nbhd}).
    We define the \emph{exponential map} at p to be
    \begin{align*}
        \exp_p:U&\to M\\
        v&\mapsto\gamma_v(1)
    \end{align*}
\end{defn}

\begin{rem}
    We scale the initial vector and can obtain
    \[\exp_p(v)=\gamma_v(1)=\gamma_{v/|v|}(|v|)\]
    This means the exponential map act on $v$ is moving forward distance $|v|$ along the geodesic with initial direction $v/|v|$.
\end{rem}

\begin{prop}
    $\exp_{p*}|_0:T_0(T_pM)\to T_pM$ is identity (we identify $T_0(T_pM)$ with $T_pM$).
\end{prop}
\begin{proof}
    We have
    \[\exp_{p*}|_0(v)=\left.\frac{\d{}}{\d{t}}\right|_{t=0}\exp_p(tv)=v.\qedhere\]
\end{proof}

\begin{cor}
    There is a ball $B_\varepsilon(0)\subset T_pM$ such that $\exp_p:B_\varepsilon(0)\to M$ is a diffeomorphism onto its image.
\end{cor}
\begin{proof}
    Since $\exp_{p*}|_0$ is identity, it is nondegenerate, the corollary follows by Inverse Function Theorem.
\end{proof}

\begin{eg}
    \begin{enumerate}[(1)]
        \item We know that the geodesics on $\mathbb{S}^n$ are great circles, hence $\exp_p$ is defined on the whole $T_pM$.
        But $\exp_p$ is not injective, since $\exp_p(0)=\exp_p(2\pi v)=p$ for unit vector $v$ in $T_pM$.
        \item Let $M=\mathbb{S}^1\times\mathbb{R}$ be the cylinder.
        We know from elementary differential geometry that the geodesics on cylinder are directrix circles, helices and generatrix lines.
        Then in local chart $(e^{2\pi it},s)\mapsto(t,s)$, we know the $\exp_p$ is not injective in the direction $(1,0)$, and injective in other directions.
    \end{enumerate}
\end{eg}

\begin{defn}\label{geodesically completeness}
    If $\exp_p$ can be defined on whole $T_pM$ for any $p\in M$, we say $M$ is \emph{geodesically complete}.
\end{defn}

We have the following important theorem.

\begin{thm}[Hopf--Rinow]\index{theorem!Hopf--Rinow}
    Let $M$ be a Riemannian manifold, the following are equivalent:
    \begin{enumerate}[(1)]
        \item $M$ is geodesically complete.
        \item $\exp_p:T_pM\to M$ is well-defined for some $p\in M$.
        \item The \emph{Heine--Borel property} holds, that is, any closed bounded set is compact on $M$.
        \item $M$ is complete as a metric space, that is, any Cauchy sequence converges.
    \end{enumerate}
\end{thm}

We will not prove Hopf--Rinow theorem here.
For a proof, one can refer to Peter Petersen's \emph{Riemannian Geometry}, 3rd ed., Theorem 5.7.1.

\section{Curvature}\label{sect of curv}

We know $\exp_{p*}|_0$ is identity, and we want to ask:
\begin{ques}
    What is $\exp_{p*}|_v:T_v(T_pM)\to T_{\exp_{p}(v)}M$?
\end{ques}

To calculate $\exp_{p*}|_v(\xi)$, we choose a line $v+s\xi$, and then
\[\exp_{p*}|_v(\xi)=\left.\frac{\d{}}{\d{s}}\right|_{s=0}\exp_p(v+s\xi)\]
Now we can introduce a family of geodesics $\gamma(t,s)=\gamma_s(t)=\exp_p(t(v+s\xi))$, and denote $\gamma(t)=\gamma(t,0)$.
Let $J_s(t)=\frac{\partial{}}{\partial{s}}\gamma(t,s)$, then $\dot{J}_s(t)=\nabla_{\dot{\gamma}_s(t)}\frac{\partial{\gamma}}{\partial{s}}$.
Since $\nabla_{\dot{\gamma}_s(t)}\frac{\partial{\gamma}}{\partial{t}}=0$, we have
\begin{align*}
    \ddot{J}_s(t)&=\nabla_{\dot{\gamma}_s(t)}\nabla_{\dot{\gamma}_s(t)}\frac{\partial{\gamma}}{\partial{s}}\\
    &=\nabla_{\dot{\gamma}_s(t)}\nabla_{\frac{\partial{\gamma}}{\partial{s}}}\frac{\partial{\gamma}}{\partial{t}}\quad\text{(torsion-freeness)}\\
    &=\nabla_{\frac{\partial{\gamma}}{\partial{t}}}\nabla_{\frac{\partial{\gamma}}{\partial{s}}}\frac{\partial{\gamma}}{\partial{t}}-\nabla_{\frac{\partial{\gamma}}{\partial{s}}}\nabla_{\frac{\partial{\gamma}}{\partial{t}}}\frac{\partial{\gamma}}{\partial{t}}.
\end{align*}
Denote
\[R\left(\frac{\partial{\gamma}}{\partial{t}},\frac{\partial{\gamma}}{\partial{s}}\right)=\nabla_{\frac{\partial{\gamma}}{\partial{s}}}\nabla_{\frac{\partial{\gamma}}{\partial{t}}}-\nabla_{\frac{\partial{\gamma}}{\partial{t}}}\nabla_{\frac{\partial{\gamma}}{\partial{s}}}+\nabla_{\left[\frac{\partial{\gamma}}{\partial t},\frac{\partial{\gamma}}{\partial{s}}\right]},\]
then we have
\[\frac{\partial^2{}}{\partial{t^2}}J_s(t)+R\left(\frac{\partial{\gamma}}{\partial{t}},\frac{\partial{\gamma}}{\partial{s}}\right)\frac{\partial{\gamma}}{\partial{t}}=0.\]
Let $s=0$, we have
\[\ddot{J}(t)+R(\dot{\gamma}(t),J(t))\dot{\gamma}(t)=0.\]
We make it into a definition.

\begin{defn}[Riemann curvature tensor]\index{curvature!Riemann curvature tensor}
    Let $R:\mathfrak{X}(M)\times\mathfrak{X}(M)\times\mathfrak{X}(M)\to\mathfrak{X}(M)$ defined by
    \[R(X,Y)Z=\nabla_Y\nabla_XZ-\nabla_X\nabla_YZ+\nabla_{[X,Y]}Z.\]
\end{defn}

We also look at Riemann curvature tensor locally.
Tedious calculation will show Riemann curvature tensor is truly tensorial.
Using a more simple notation $\partial_i=\partial{}/\partial{x^i}$, we have
\begin{align*}
    R(\partial_i,\partial_j)\partial_k&=\nabla_{\partial_j}\nabla_{\partial_i}\partial_k-\nabla_{\partial_i}\nabla_{\partial_j}\partial_k\\
    &=\nabla_{\partial_j}(\Gamma^l_{ik}\partial_l)-\nabla_{\partial_i}(\Gamma^l_{jk}\partial_l),
\end{align*}
and
\begin{align*}
    R_{ijk}^l&=(\partial_j\Gamma^l_{ik}-\partial_i\Gamma^l_{jk}+\Gamma^m_{ik}\Gamma^l_{jm}-\Gamma^m_{jk}\Gamma^l_{im})\partial_l\\
    &=\partial^2g+\partial g*\partial g
\end{align*}
We also define $R_{ijkl}=R_{ijk}^mg_{ml}$, or $R(X,Y,Z,W)=g(R(X,Y)Z,W)$.

\begin{eg}\index{flat}
    $(\mathbb{R},\delta)$ has $R\equiv 0$.
    Any metric admits zero curvature is call \emph{flat}.
\end{eg}

\begin{prop}
    Riemann curvature tensor has following symmetric properties: For $X,Y,Z,W\in\mathfrak{X}(M)$, we have
    \begin{enumerate}[(1)]
        \item $R(X,Y,Z,W)=-R(Y,X,Z,W)=-R(X,Y,W,Z)=R(Z,W,X,Y)$;
        \item $R(X,Y,Z,W)+R(Y,Z,X,W)+R(Z,X,Y,W)=0$ (First Bianchi Identity).
    \end{enumerate}
\end{prop}
\begin{proof}
    Tedious calculation.
\end{proof}

\begin{defn}[Sectional curvature]\index{curvature!sectoinal}
    Let $p\in M$, $\pi\subset T_pM$ be a $2$-plane, $\pi=\Span\{X,Y\}$.
    Then define the sectional curvature at $p$ of $\pi$ as
    \[K_p(\pi):=\frac{R(X,Y,X,Y)}{|X|^2|Y|^2-\langle X,Y\rangle^2}.\]
\end{defn}

\begin{rem}
    One can show sectional curvature does not depend on the choice of basis.
    A proof can be found in Manfredo do Carmo's \emph{Riemannian Geometry}, Proposition 3.1.
\end{rem}

\begin{prop}
    Let $M$ be a Riemannian manifold.
    $R(X,Y,Z,W)$ as a $(0,4)$-tensor is determined by all $K_p(\pi)$.
\end{prop}
For a proof, see Wu Hung--Hsi et. al., \emph{Introduction to Riemannian Geometry}, Chapter 2 Lemma 2.

We mention one little observation.
If all sectional curvature at $p$ is constant $K_p$, then
\[R_p(X,Y,Z,W)=K_p(\langle X,Z\rangle\langle Y,W\rangle-\langle X,W\rangle\langle Y,Z\rangle)\]

For a theorem about constant sectoinal curvature, we mention here Schur's Theorem.
\begin{thm}[Schur]\index{theorem!Schur}
    Let $(M^n,g)$ be a Riemannian manifold, $n\geq 3$.
    If $K_p(\pi)$ is independent of $\pi\subset T_pM$ for any $p\in M$, then $M$ has constant sectional curvature.
\end{thm}

We define another two important curvature.
\begin{defn}[Ricci curvature]\index{curvature!Ricci}
    Let $(M^n,g)$ be a Riemannian manifold.
    Define \emph{Ricci curvature tensor} $\Ric:\mathfrak{X}(M)\to\mathfrak{X}(M)$ as
    \[\Ric_p(X)=\sum_{i=1}^nR_p(e_i,X)e_i,\]
    where $\{e_i\}$ is an orthonormal frame around $p$.
    It's easy to check the definition is independent from the choice of orthonormal frame, and $\Ric$ is self-adjoint.
\end{defn}

\begin{defn}[Scalar curvature]\index{curvature!scalar}
    Let $(M^n,g)$ be a Riemannian manifold.
    We define \emph{Scalar curvature} $\Scal\in C^\infty(M)$ as
    \[\Scal(p)=\sum_{i=1}^n\langle\Ric_p(e_i),e_i(p)\rangle\]
    for an orthonormal frame $\{e_i\}$ around $p$.
\end{defn}

\begin{defn}
    Let $(M,g)$ be a Riemannian manifold, if $\Ric=\lambda(p)g$ for a $\lambda\in C^\infty(M)$, we call $M$ an \emph{Einstein manifold}.
\end{defn}

We also mention here
\begin{thm}[Schur]\index{theorem!Schur}
    Let $M^n$ be an Einstein manifold with $n\geq 3$, then $M$ has constant scalar curvature.    
\end{thm}

\section{Jacobi Fields}

\begin{defn}\index{Jacobi field}
    Let $\gamma$ be a geodesic, a vector field $J$ along $\gamma$ is called a \emph{Jacobi field} if $\ddot{J}+R(\dot{\gamma},J)\dot{\gamma}=0$.
\end{defn}

Let $\{e_i(t)\}$ be a parallel orthonormal frame along $\gamma$, $J(t)=\sum_if_i(t)e_i(t)$.
Define $a_{ij}(t)=\langle R(\dot{\gamma}(t),e_i(t))\dot{\gamma}(t),e_j(t)\rangle$, then the qeuation for Jacobi field is equivalent to
\[\ddot{f}_i(t)+\sum_ja_ij(t)f_j(t)=0,\ i=1,2,\cdots,n.\]
By ODE theory, given $f_i(0),\dot{f}_i(0)$, $i=1,2,\cdots,n$, the $f_i(t)$'s are uniquely determined.
Translating into language of Jacobi field, we have a Jacobi field is uniquely determined by $J(0)$ and $\dot{J}(0)$.

\begin{symb}
    Let $\gamma$ be a geodesic on a Riemannian manifold $M$, the vector space of Jacobi fields along $\gamma$ is denoted by $\mathscr{J}(\gamma)$.
\end{symb}

Above discussion can be summarized as the following proposition.

\begin{prop}
    Let $M^n$ be a Riemannian manifold, $\gamma$ be a geodesic, then $\dim\mathscr{J}(\gamma)=2n$.
\end{prop}

The next proposition shows only the Jacobi field at tangential direction are interesting.
\begin{prop}\label{decomposition of Jacobi fields}\index{Jacobi field!normal}
    Let $J$ be a Jacobi field along $\gamma$, we have the decomposition $J(t)=J^\perp(t)+(at+b)\dot{\gamma}(t)$, where $J^\perp(t)\perp\dot{\gamma}(t)$.
    If a Jacobi field is perpendicular to $\gamma$, we call it a \emph{normal Jacobi field}
\end{prop}
\begin{proof}
    We have
    \begin{align*}
        \frac{\d^2{}}{\d{t^2}}\langle J(t),\dot{\gamma}(t)\rangle&=\langle\ddot{J}(t),\dot{\gamma}(t)\rangle\\
        &=-\langle R(\dot{\gamma},J)\dot{\gamma},\dot{\gamma}\rangle\\
        &=0.\qedhere
    \end{align*}
\end{proof}

We have the following Gauss Lemma.
\begin{prop}[Gauss Lemma]\index{lemma!Gauss}
    $\langle\exp_{p*}|_v(\xi),\dot{\gamma}_v(1)\rangle=\langle\xi,v\rangle$.
\end{prop}
\begin{proof}
    Let one-parameter geodesic family $\gamma(t,s)=\exp_p(t(v+s\xi))$, then using the calculation in the beginning of section~\ref{sect of curv}, we have
    \begin{align*}
        J(t)=\left.\frac{\partial{}}{\partial{s}}\right|_{s=0}\gamma(t,s)&=\exp_{p*}|_{tv}(t\xi)\\
        &=t\exp_{p*}|_{tv}(\xi).
    \end{align*}
    Moreover, we have $\langle J(t),\dot{\gamma}_v(t)\rangle=at+b$ (Proposition~\ref{decomposition of Jacobi fields}),
    \begin{align*}
        a&=\left.\frac{\d{}}{\d{t}}\right|_{t=0}\langle J(t),\dot{\gamma}_v(t)\rangle=\langle\dot{J}(0),\dot{\gamma}_v(0)\rangle+\langle J(0),\nabla_v\dot{\gamma}_v(t)\rangle\\
        &=\langle\dot{J}(0),v\rangle=\left\langle\left.\left(\exp_{p*}|_{tv}(\xi)+t\frac{\d{}}{\d{t}}\exp_{p*}|_{tv}(\xi)\right)\right|_{t=0},v\right\rangle\\
        &=\langle\exp_{p*}|_0(\xi),v\rangle=\langle\xi,v\rangle,
    \end{align*}
    and $b=\langle J(0),v\rangle=0$.
    But we have
    \[t\langle\exp_{p*}|_{tv}(\xi),\dot{\gamma}_v(t)\rangle=\langle J(t),\dot{\gamma}_v(t)\rangle=\langle\xi,v\rangle t.\]
    Let $t=1$ we obtain the conclusion.
\end{proof}

\begin{eg}
    We calculate the Jacobi field on Riemannian manifolds admit a constant sectional curvature.
    By scaling the metric, we can assume $K=0,1,-1$.
    The corresponding simply connected complete Riemannian manifolds are called \emph{space forms}\index{space form}, which are $\mathbb{R}^n,\mathbb{S}^n,\mathbb{H}^n$.
    The Jacobi field equation is
    \[\begin{cases}
        \ddot{J}(t)+KJ(t)=0,\\
        J(0)=0,\dot{J}(0)=\xi.
    \end{cases}\]
    Let $\xi(t)$ be the parallel transport along a geodesic with $\xi(0)=\xi$, then solving the equation by eigenvalue method, we obtain
    \[J(t)=\begin{cases}
        t\xi(t), & K=0,\\
        \sin(t)\xi(t), & K=1,\\
        \sinh(t)\xi(t), & K=-1.
    \end{cases}\]
\end{eg}

\section{Some Local Charts}
In this section, we adopt the traditional terminology ``coordinate'' to mean chart.

First we introduce the geodesic normal coordinate.\index{geodesic normal coordinate}
Given a Riemannian manifold $(M,g)$ and $p\in M$, there exists an $\varepsilon>0$ such that $\exp_p:B_\varepsilon(0)\to\exp_p(B_\varepsilon(0))=:B_\varepsilon(p)$ is an diffeomorphism.
Let $\{e_i\}$ be an orthonormal basis of Euclidean space $(T_pM,g_p)$, $\{\alpha^i\}$ be the dual basis of $\{e_i\}$, then we construct the \emph{geodesic normal coordinate}
\[q\in B_\varepsilon(p)\mapsto(\alpha^1(\exp_p^{-1}(q)),\cdots,\alpha^n(\exp_p^{-1}(q))).\]

\begin{prop}
    Under geodesic normal coordinate, we have
    \[g_{ij}(p)=\delta_{ij},\ \Gamma^k_{ij}(p)=0\]
\end{prop}
\begin{proof}
    Since $\exp_p$ is a diffeomorphism, we have $\left.\frac{\partial{}}{\partial{x^i}}\right|_p=\exp_{p*}|_0(e_i)=e_i$, hence $g_{ij}=g_p(e_i,e_j)=\delta_{ij}$.
    Moreover, let $x(t)=ty$ for $y\in T_pM\backslash\{0\}$, then $x(t)$ is the coordinate of some geodesic in $B_\varepsilon(p)$, thus it satisfies the equation
    \[\ddot{x}^k(t)+\Gamma^k_{ij}(x(t))\dot{x}^i(t)\dot{x}^j(t)=0.\]
    Since $\ddot{x}^k(t)=0$, $\dot{x}^i(t)=y^i\neq 0$, we must have $\Gamma^k_{ij}(ty)=0$.
    Let $y\to 0$ and we obtain the conclusion.
\end{proof}

Next we introduce the geodesic polar coordinate.\index{geodesic polar coordinate}
Let $(r,\theta^1,\cdots,\theta^{n-1})$ be the polar coordinate on Euclidean space $(T_pM,g_p)$, and we define the \emph{geodesic polar coordinate} by
\[q\in B_\varepsilon(p)\backslash\{p\}\mapsto(r(\exp_p^{-1}(q)),\theta^1(\exp_p^{-1}(q)),\cdots,\theta^{n-1}(\exp_p^{-1}(q))).\]

\begin{prop}
    Under geodesic polar coordinate, we have
    \[g\left(\frac{\partial{}}{\partial{r}},\frac{\partial{}}{\partial{r}}\right)=1,\ g\left(\frac{\partial{}}{\partial{r}},\frac{\partial{}}{\partial{\theta^i}}\right)=0\]
\end{prop}
\begin{proof}
    To make things clear, we write the inverse of geodesic polar coordinate as
    \[F:(r,\omega)\mapsto\exp_p(r\omega)\]
    for $r\in(0,+\infty),\omega\in\mathbb{S}^{n-1}$.
    Then we use $\partial_r,\partial_{\theta^1},\cdots,\partial_{\theta^{n-1}}$ to denote the tangent vectors in $(0,+\infty)\times\mathbb{S}^{n-1}$, we have
    \begin{align*}
        &\frac{\partial{}}{\partial{r}}=F_*(\partial_r)\\
        &\frac{\partial{}}{\partial{\theta^i}}=F_*(\partial_{\theta^i}),\ i=1,\cdots,n.
    \end{align*}
    First we know $\partial_r$ is the tangent vector of direction $r\omega$, hence $\partial{}/\partial{r}$ is the tangent vector of a unit-speed radial geodesic, that is
    \[g\left(\frac{\partial{}}{\partial{r}},\frac{\partial{}}{\partial{r}}\right)=1.\]
    Moreover, we have
    \begin{align*}
        \frac{\partial{}}{\partial{r}}g\left(\frac{\partial{}}{\partial{r}},\frac{\partial{}}{\partial{\theta^i}}\right)&=g\left(\nabla_{\frac{\partial{}}{\partial{r}}}\frac{\partial{}}{\partial{r}},\frac{\partial{}}{\partial{\theta^i}}\right)+g\left(\frac{\partial{}}{\partial{r}},\nabla_{\frac{\partial{}}{\partial{r}}}\frac{\partial{}}{\partial{\theta^i}}\right)\\
        &=g\left(\frac{\partial{}}{\partial{r}},\nabla_{\frac{\partial{}}{\partial{r}}}\frac{\partial{}}{\partial{\theta^i}}\right)\\
        &=g\left(\frac{\partial{}}{\partial{r}},\nabla_{\frac{\partial{}}{\partial{\theta^i}}}\frac{\partial{}}{\partial{r}}\right)\quad\text{(torsion-freeness)}\\
        &=\frac{1}{2}\frac{\partial{}}{\partial{\theta^i}}g\left(\frac{\partial{}}{\partial{r}},\frac{\partial{}}{\partial{r}}\right)\\
        &=0,
    \end{align*}
    hence $g\left(\frac{\partial{}}{\partial{r}},\frac{\partial{}}{\partial{\theta^i}}\right)$ is constant.
    But if we let $r\to 0$, we have $\partial{}/\partial{\theta^i}\to 0$, therefore
    \[g\left(\frac{\partial{}}{\partial{r}},\frac{\partial{}}{\partial{\theta^i}}\right)=0\qedhere\]
\end{proof}

\begin{cor}
    Under geodesic polar coordinate, the metric tensor has local expression
    \[g=\d{r^2}+g_{ij}(r,\theta)\d{\theta^i}\otimes\d{\theta^j}\]
\end{cor}

As an application, we prove geodesics are locally length-minimizing.
\begin{prop}\label{locally length-min}\index{geodesic!locally length minimizing}
    Let $\gamma:[0,1]\to M$ be a geodesic in $B_\varepsilon(p)$, $\tilde{\gamma}:[0,1]\to M$ be any curve in $B_\varepsilon(p)$ with $\gamma(0)=\tilde{\gamma}(0)=p$, $\gamma(1)=\tilde{\gamma}(1)=q$.
    Then $L(\gamma)\leq L(\tilde{\gamma})$.
\end{prop}
\begin{proof}
    Let $q=\exp_p(v)$, $\varphi$ be the geodesic polar coordinate, then we have
    \[\gamma(t)=(tr_0,\omega_0),\ \tilde{\gamma}(t)=(r(t),\omega(t))\]
    such that $\omega_0,\omega(t)\in\mathbb{S}^{n-1}$, $r(1)=r_0$.
    Therefore
    \begin{align*}
        L(\gamma)&=\int_0^1g_{\gamma(t)}(\dot{\gamma}(t),\dot{\gamma}(t))\d{t}\\
        &=\int_0^1|v|\d{t}=r_0\\
        L(\tilde{\gamma})&=\int_0^1(|\dot{r}(t)|^2+g_{ij}\dot{\theta}^i(t)\dot{\theta}^j(t))^{1/2}\d{t}\\
        &\geq\int_0^1|\dot{r}(t)|\d{t}\\
        &\geq\int_0^1\dot{r}(t)=r_0\qedhere
    \end{align*}
\end{proof}

\section{Conjugate Points}

\begin{defn}[Conjugate points]\index{conjugate point}
    If $\exp_{p*}|_{\dot{\gamma}_v(t_0)}$ is degenerate at $\dot{\gamma}_v(t_0)$, then we call $\gamma_v(t_0)$ a \emph{conjugate point} of $p$ along $\gamma$.
\end{defn}

\begin{prop}
    We have $\exp_p$ is degenerate at $\gamma(t_0)$ if and only if there is a Jacobi field $J$ not identically equal to $0$ such that $J(0)=J(t_0)=0$.
\end{prop}
\begin{proof}
    $\exp_{p*}|_{\dot{\gamma}_v(t_0)}(\xi)=0$ if and only if $J(t)=\left.\frac{\partial{}}{\partial{s}}\right|_{s=0}\exp_p(t(v+s\xi))$ satisfies $J(0)=J(t)=0$.
\end{proof}

\begin{rem}
    This proposition shows conjugate is symmetric.
\end{rem}