\chapter{Some Background on Tensors}
We now supply some background materials on tensors.
We first discuss the covariant derivative of tensors.
Then we define Hessian, Laplacian and gradient of a function.
Finally, we define the norm of symmetric $(0,2)$-tensors.

For simplicity, we consider only covariant tensors.

\begin{defn}
    Let $T$ be a $(0,r)$-tensor ($r\geq 0$), then we define the \textbf{covariant derivative} of $T$ as
    \[\nabla_XT(Y_1,\cdots,Y_r)=XT(Y_1,\cdots,Y_r)-\sum_{i=1}^rT(Y_1,\cdots,\nabla_XY_i,\cdots,Y_r)\]
    for any vector fields $Y_1,\cdots,Y_r$.
    Notice that if $T$ is a function (i.e., $(0,0)$-tensor), the minus summation term does not exist.
    Moreover, we define the \textbf{covariant differential} of $T$ as
    \[\nabla T(Y_1,\cdots,Y_r,X)=\nabla_XT(Y_1,\cdots,Y_r).\]
\end{defn}

One can check $\nabla T$ is indeed tensorial, and satisfies Leibniz rule, that is
\[\nabla(T\otimes S)=(\nabla T)\otimes S+T\otimes(\nabla S).\]

We look at covariant derivative locally.
It's known that $(0,r)$-tensors have a basis
\[\{\d{x^{i_1}}\otimes\cdots\otimes\d{x^{i_r}}\}_{1\leq i_1,\cdots,i_r\leq n},\]
(We will not prove here, please refer to any book on differentiable manifolds)
so we let
\[T=T_{i_1\cdots i_r}\d{x^{i_1}}\otimes\cdots\otimes\d{x^{i_r}},\]
and write
\[\nabla_{\partial_k}T=T_{i_1\cdots i_r;k}\d{x^{i_1}}\otimes\cdots\otimes\d{x^{i_r}}.\]
Let's compute the coefficient. First we compute
\[\nabla_{\partial_i}\d{x^j}=a_k\d{x^k},\]
then
\begin{align*}
    a_k&=\nabla_{\partial_i}\d{x^j}(\partial_k)\\
    &=-\d{x^j}(\nabla_{\partial_i}\partial_k)\\
    &=-\d{x^j}(\Gamma^l_{ik}\partial_l)\\
    &=-\Gamma^j_{ik},
\end{align*}
that is,
\[\nabla_{\partial_i}d{x^j}=-\Gamma^j_{ik}d{x^k}.\]
Using this, we can compute
\begin{align*}
    &\nabla_{\partial_k}(T_{i_1\cdots i_r}\d{x^{i_1}}\otimes\cdots\otimes\d{x^{i_r}})\\
    =&\partial_kT_{i_1\cdots i_r}\d{x^{i_1}}\otimes\cdots\otimes\d{x^{i_r}}+T_{i_1\cdots i_r}\sum_{l=1}^r\d{x^{i_1}}\otimes\cdots\otimes(\nabla_{\partial_k}\d{x^{i_j}})\otimes\cdots\otimes\d{x^{i_r}}\\
    =&\partial_kT_{i_1\cdots i_r}\d{x^{i_1}}\otimes\cdots\otimes\d{x^{i_r}}+T_{i_1\cdots i_r}\sum_{l=1}^r\d{x^{i_1}}\otimes\cdots\otimes(-\Gamma^{i_j}_{kp}\d{x^p})\otimes\cdots\otimes\d{x^{i_r}}\\
    =&\partial_kT_{i_1\cdots i_r}\d{x^{i_1}}\otimes\cdots\otimes\d{x^{i_r}}-\sum_{l=1}^rT_{i_1\cdots i_{l-1}pi_{l+1}\cdots i_r}\Gamma^p_{ki_l}\d{x^{i_1}}\otimes\cdots\otimes\d{x^{i_r}},
\end{align*}
hence we have
\[T_{i_1\cdots i_r;k}=\partial_kT_{i_1\cdots i_r}-\sum_{l=1}^rT_{i_1\cdots i_{l-1}pi_{l+1}\cdots i_r}\Gamma^p_{ki_l}.\]

Moreover, we can consider second covariant derivative.
\begin{symb}
    We use the symbol $\nabla^2_{X,Y}T$ to denote the tensor 
    \[\nabla^2_{X,Y}T(Y_1,\cdots,Y_r)=\nabla(\nabla T)(Y_1,\cdots,Y_r,Y,X).\]
\end{symb}

\begin{prop}
    We have
    \begin{equation}
        \nabla^2_{X,Y}T=\nabla_X(\nabla_YT)-\nabla_{\nabla_XY}T.\label{2nd cov}
    \end{equation}
\end{prop}
\begin{proof}
    We have
    \begin{align*}
        &\nabla^2_{X,Y}T(Y_1,\cdots,Y_r)\\
        =&\nabla(\nabla T)(Y_1,\cdots,Y_r,Y,X)\\
        =&\nabla_X(\nabla T)(Y_1,\cdots,Y_r,Y)\\
        =&X(\nabla T)(Y_1,\cdots,Y_r,Y)-\sum_{i=1}^r(\nabla T)(Y_1,\cdots,\nabla_XY_i,\cdots,Y_r,Y)\\
        &-(\nabla T)(Y_1,\cdots,Y_r,\nabla_XY)\\
        =&X(\nabla_YT)(Y_1,\cdots,Y_r)-\sum_{i=1}^r(\nabla_YT)(Y_1,\cdots,\nabla_XY_i,\cdots,Y_r)\\
        &-(\nabla_{\nabla_XY}T)(Y_1,\cdots,Y_r)\\
        =&\nabla_X(\nabla_YT)(Y_1,\cdots,Y_r)-(\nabla_{\nabla_XY}T)(Y_1,\cdots,Y_r).\qedhere
    \end{align*}
\end{proof}

Next we discuss curvature of tensors.

\begin{defn}[Curvature operator]
    Let $X,Y$ be vector fields, define an endomorphism $R(X,Y)$ of $(0,r)$-tensors as
    \[R(X,Y)T=\nabla_Y\nabla_XT-\nabla_X\nabla_YT+\nabla_{[X,Y]}T.\]
\end{defn}

\begin{prop}[Ricci identity]
    For a $(0,r)$-tensor $T$ and vector fields $X,Y$, we have
    \[(\nabla^2_{Y,X}-\nabla^2_{X,Y})T=R(X,Y)T\]
\end{prop}
\begin{proof}
    Using equation~\ref{2nd cov}, we have
    \begin{align*}
        (\nabla^2_{Y,X}-\nabla^2_{X,Y})T&=\nabla_Y\nabla_XT-\nabla_{\nabla_YX}T-\nabla_X\nabla_YT+\nabla_{\nabla_XY}T\\
        &=\nabla_Y\nabla_XT-\nabla_X\nabla_YT+\nabla_{(\nabla_XY-\nabla_YX)}T\\
        &=\nabla_Y\nabla_XT-\nabla_X\nabla_YT+\nabla_{[X,Y]}T\quad\text{(torsion-freeness)}\\
        &=R(X,Y)T.\qedhere
    \end{align*}
\end{proof}

Now we introduce some differential operators.

\begin{defn}
    If $S$ is a symmetric $(0,2)$-tensor on  a Riemannian manifold $(M,g)$, we define the \textbf{trace} of $S$ to be
    \[\tr_g(S)|_p=\sum_{i=1}^nS_p(e_i,e_i),\ \forall p\in M\]
    for any orthonormal basis $\{e_i\}$ of $T_pM$.
    It can be checked that the definition does not depend on the choice of orthonormal basis.
    By this definition, trace is a smooth function on $M$.
\end{defn}

If $S=S_{ij}\d{x^i}\otimes\d{x^j}$, then one can check that $\tr_g(S)=g^{ij}S_{ij}$.

\begin{defn}
    Let $(M,g)$ be a Riemannian manifold, $f\in C^\infty(M)$.
    Define the \textbf{Hessian} of $f$ to be a $(0,2)$-tensor $\nabla^2f$.
    Ricci identity shows $\nabla^2f$ is symmetric, hence we define its trace to be the \textbf{Laplacian} of $f$, denoted by $\Delta f$.
\end{defn}

\begin{prop}
    Let $(M,g)$ be a Riemannian manifold, $f\in C^\infty(M)$.
    Then we have
    \[\nabla^2f(X,Y)=Y(Xf)-(\nabla_YX)(f).\]
\end{prop}
\begin{proof}
    This is just the equation~\ref{2nd cov}.
\end{proof}

We introduce the gradient of a function.
\begin{defn}
    Let $(M,g)$ be a Riemannian manifold, $f\in C^\infty(M)$.
    Then we define the \textbf{gradient} of $f$ to be the vector field $\nabla f$ defined by $g(\nabla f,X)=Xf$ for any $X\in\mathfrak{X}(M)$.
\end{defn}

We now give the local expression of $\nabla f$ (and show the existence of $\nabla f$ as a by-product).
Let $\nabla f=f^i\partial_i$, then we have
\[g(f,\partial_j)=f^ig_{ij}=\partial_jf.\]
Then use $g^{ij}$ to denote the inverse matrix of $g_{ij}$, we have
\[f^i=g^{ij}\partial_jf,\]
that is,
\[\nabla f=g^{ij}\frac{\partial f}{\partial x^{j}}\frac{\partial{}}{\partial{x^i}}.\]

Finally we define the norm of a symmetric $(0,2)$-tensor.

\begin{defn}
    Let $S$ be a symmetric $(0,2)$-tensor.
    At $p\in M$, let $\{e_i\}$ be an orthonormal basis, $\{f^i\}$ be its dual basis.
    Let $S=S_{ij}f^i\otimes f^j$, then define the norm of $S$ to be
    \[|S|^2:=\sum_{i,j}|S_{ij}|^2.\]
\end{defn}

We also write the norm of $S$ in general coordinates.
One can check under general $g$ we have
\[|S|^2=g^{il}g^{jk}S_{ij}S_{kl},\]
for $S=S_{ij}\d{x^i}\otimes\d{x^j}$.