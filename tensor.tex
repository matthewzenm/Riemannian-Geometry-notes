\chapter{Some Background on Tensors}
We now supply some background materials on tensors.
% We first discuss the covariant derivative of tensors.
% Then we define Hessian, Laplacian and gradient of a function.
% Finally, we define the norm of symmetric $(0,2)$-tensors.

% For simplicity, we consider only covariant tensors.

\section{Basic Notions}

First we introduce the notion of tensors and differential forms.
In this section, we will not give proof of any proposition, please refer to any book on differentiable manifolds for proofs.
\begin{defn}\index{tensor}
    Let $M$ be a differentiable manifold.
    An \emph{$(r,s)$-tensor} $T$ is a $C^\infty(M)$-multi\-linear map
    \[T:\underbrace{\mathfrak{X}(M)^*\times\cdots\times\mathfrak{X}(M)^*}_{r}\times\underbrace{\mathfrak{X}(M)\times\cdots\times\mathfrak{X}(M)}_{s}\to C^\infty(M),\]
    where $\mathfrak{X}(M)^*$ denotes the dual module of $\mathfrak{X}(M)$ over $C^\infty(M)$.
    $(0,s)$-tensors are called \emph{covariant tensors}, and $(r,0)$-tensors are called \emph{contravariant tensors}.
\end{defn}

\begin{rem}
    We often define $(0,0)$-tensors to be smooth functions, and identify $(1,0)$-tensors with vector fields.
\end{rem}

\begin{defn}\index{tensor product}
    Let $\omega,\eta\in\mathfrak{X}(M)^*$.
    The \emph{tensor product} of $\omega,\eta$, denoted by $\omega\otimes\eta$, is defined as
    \begin{align*}
        \omega\otimes\eta:\mathfrak{X}(M)\times\mathfrak{X}(M)&\to C^\infty(M)\\
        (X,Y)&\mapsto\omega(X)\cdot\eta(Y).
    \end{align*}
    Clearly tensor product is associative, and distributive to addition.
    We can also define tensor product for contravariant tensors.
\end{defn}

\begin{prop}
    Let $\mathfrak{T}^{r,s}(M)$ be the $C^\infty(M^n)$ module of $(r,s)$-tensors.
    Then on a local chart $(U,\varphi)$, $\mathfrak{T}^{r,s}|_U$ is free with basis
    \[\left\{\frac{\partial{}}{\partial{x^{i_1}}}\otimes\cdots\otimes\frac{\partial{}}{\partial{x^{i_r}}}\otimes\d{x^{j_1}}\otimes\cdots\otimes\d{x^{j_s}}\right\}\]
    for all indices $1\leq i_1,\cdots,i_r\leq n$, $1\leq j_1,\cdots,j_s\leq n$, where
    \[\d{x^i}\left(\frac{\partial{}}{\partial{x^j}}\right)=\delta_{ij}.\]
\end{prop}

\begin{defn}\index{differential form}
    A \emph{differential form} on $M$ of order $r$ is a skew-symmetric $(0,r)$-tensor.
    The space of differential form of order $r$ is denoted by $\bigwedge\nolimits^rM$.
\end{defn}

\begin{prop}
    There is a map
    $\pi:\mathfrak{T}^r(M)\to\bigwedge\nolimits^rM$ as a quotient map of $C^\infty(M)$ modules given by
    \[\omega(X_1,\cdots,X_r)\mapsto\frac{1}{r!}\sum_{\sigma\in S_r}\operatorname{sgn}(\sigma)\omega\left(X_{\sigma(1)},\cdots,X_{\sigma(r)}\right)\]
    for any $(0,r)$-tensor $\omega$ and $X_i\in\mathfrak{X}(M)$, $i=1,\cdots,r$.
\end{prop}

\begin{defn}\index{wedge product}
    We have a bilinear map $\wedge:\bigwedge\nolimits^r(M)\times\bigwedge\nolimits^s(M)\to\bigwedge\nolimits^{r+s}(M)$ defined by the commutative diagram
    \[\begin{tikzcd}
        \mathfrak{T}^r(M)\times\mathfrak{T}^s(M) \ar[r,"\otimes"] \ar[d,"\pi\times\pi"] & \mathfrak{T}^{r+s}(M) \ar[d,"\pi"] \\
        \bigwedge\nolimits^r(M)\times\bigwedge\nolimits^s(M) \ar[r,"\wedge"] & \bigwedge\nolimits^{r+s}(M)
    \end{tikzcd}\]
\end{defn}

\begin{prop}
    Under a local chart $(U,\varphi)$, $\left.\bigwedge\nolimits^r(M)\right|_U$ has a basis
    \[\{\d{x^{i_1}}\wedge\cdots\wedge\d{x^{i_r}}\}\]
    for indices $1\leq i_1<\cdots<i_r\leq n$.
\end{prop}

\begin{defn}\index{exterior differential}
    We define the \emph{exterior differential} of a form $\omega$ by writing
    \[\omega=\omega_{i_1\cdots i_r}\d{x^{i_1}}\wedge\cdots\wedge\d{x^{i_r}}\]
    on a local chart and define
    \begin{align*}
        \d\omega&:=\d{\omega_{i_1\cdots i_r}}\wedge\d{x^{i_1}}\wedge\cdots\wedge\d{x^{i_r}}\\
        &:=\frac{\partial\omega_{i_1\cdots i_r}}{\partial{x^i}}\d{x^i}\wedge\d{x^{i_1}}\wedge\cdots\wedge\d{x^{i_r}}.
    \end{align*}
    One can check this definition does not depend on the choice of local chart.
\end{defn}

\section{Musical Isomorphisms}\index{musical isomorphism}
Let $(M^n,g)$ be a Riemannian manifold.
For simplicity, we only discuss the musical isomorphisms of $(1,0)$- and $(0,1)$-tensors.

\begin{defn}\index{lowering index}
    Let $X$ be a vector field (i.e.\ $(1,0)$-tensor), and on a local chart we have $X=X^i\partial_i$.
    Then we define a $(0,1)$-tensor $X^\flat$ by
    \[X^\flat_j=g_{ij}X^i.\]
    One can check this definition does not depend on the choice of local chart.
    Then $X\mapsto X^\flat$ gives an isomorphism of $\mathfrak{X}(M)\to\mathfrak{X}(M)^*$.
\end{defn}

\begin{defn}\index{raising index}
    Let $\omega$ be a $1$-form, we define $\omega^\sharp$ to be the vector field such that for any $1$-form $\eta$, we have
    \[\eta(\omega^\sharp)=\omega(\eta^\flat).\]
    Then $\omega\mapsto\omega^\sharp$ gives an isomorphism of $\mathfrak{X}(M)^*\to\mathfrak{X}(M)$.
\end{defn}

Locally, if $\omega=\omega_i\d{x^i}$, we have $(\omega^\sharp)^j=g^{ij}\omega_i$.

Notice that the indices of $X^\flat$ are lowered and of $\omega^\sharp$ are raised, this explains why we use musical notation to define these isomorphisms.

Using raising index, we can define the gradient of a function.
\begin{defn}\index{gradient}
    Let $f\in C^\infty(M)$.
    The \emph{gradient} of $f$ is defined as $\nabla f:=(\d{f})^\sharp$.
\end{defn}

By definition, $\nabla f$ has local expression
\[\nabla f=g^{ij}\frac{\partial f}{\partial{x^i}}\frac{\partial{}}{\partial{x^j}}.\]

\section{Contraction and Trace}

\begin{defn}\index{contraction}
    Let $T$ be a $(r,s)$-tensor, by \emph{contract $i,j$ indices of $T$} ($1\leq i\leq r$, $1\leq j\leq s$) we mean a tensor $\tr_{ij}T$ defined by
    \begin{align*}
        &(\tr_{ij}T)_p(\omega_1,\cdots,\widehat{\omega_i},\cdots,\omega_r,X_1,\cdots,\widehat{X_j},\cdots,X_s)\\
        =&\sum_{k=1}^nT_p(\omega_1,\cdots,\omega_{i-1},\eta_k,\omega_{i+1},\cdots,\omega_r,X_1,\cdots,X_{j-1},e_k,X_{j+1},\cdots,X_s)
    \end{align*}
    with $\{e_k\}_{k=1}^n$ be an orthonormal basis of $T_pM$ and $\{\eta_k\}_{k=1}^n$ be the dual basis.
\end{defn}

Respective to a local chart, if the tensor has local expression
\[T=T_{\mu_1\cdots\mu_s}^{\nu_1\cdots\nu_r}\frac{\partial{}}{\partial{x^{\nu_1}}}\otimes\cdots\otimes\frac{\partial{}}{\partial{x^{\nu_r}}}\otimes\d{x^{\mu_1}}\otimes\cdots\otimes\d{x^{\mu_s}},\]
then $\tr_{ij}T$ has local expression
\[(\tr_{ij}T)_{\mu_1\cdots\widehat{\mu_{j}}\cdots\mu_s}^{\nu_1\cdots\widehat{\nu_{i}}\cdots\nu_r}=T_{\mu_1\cdots\mu_{j-1}m\mu_{j+1}\cdots\mu_s}^{\nu_1\cdots\nu_{i-1}m\nu_{i+1}\cdots\nu_r}\]

\begin{defn}\index{trace}
    If $L$ is a $(1,1)$-tensor, we define its \emph{trace} by $\tr{L}=\tr_{11}L$.
\end{defn}

We know that a symmetric $(0,2)$-tensor $S$ (i.e.\ a bilinear form) is one-to-one corresponding to a $(1,1)$-tensor $L$ (i.e.\ a linear transformation), the correspondence is given by
\[S(x,x)=g(x,L(x))\]
and on a local chart
\[L^i_j=g^{ik}S_{jk}.\]
Then we have the definition

\begin{defn}\index{trace!of bilinear form}
    The trace of a symmetric $(0,2)$-tensor is defined as the trace of its corresponding $(1,1)$-tensor.
\end{defn}

We look at the trace of symmetric $(0,2)$-tensor locally.
We have
\[\tr{S}=L^i_i=g^{ij}S_{ij}.\]

We use trace to define the norm of a symmetric $(0,2)$-tensor.
\begin{defn}\index{norm of bilinear form}
    Let $S$ be a symmetric $(0,2)$-tensor, with corresponding linear transformation $L$.
    Then we define its norm by $|S|^2:=\tr(L\circ L)$.
\end{defn}

Locally, we have
\[(L\circ L)^i_j=L^i_kL^k_j\]
then contract the indices we have
\[\tr(L\circ L)=L^i_jL^j_i=(g^{ik}S_{kj})(g^{jl}S_{li})=g^{ik}g^{jl}S_{li}S_{kj}.\]

\section{Covariant Derivative}

\begin{defn}\index{covariant derivative!of tensor}
    Let $T$ be a $(0,r)$-tensor ($r\geq 0$), then we define the \emph{covariant derivative} of $T$ as
    \[\nabla_XT(Y_1,\cdots,Y_r)=XT(Y_1,\cdots,Y_r)-\sum_{i=1}^rT(Y_1,\cdots,\nabla_XY_i,\cdots,Y_r)\]
    for any vector fields $Y_1,\cdots,Y_r$.
    Notice that if $T$ is a function (i.e., $(0,0)$-tensor), the minus summation term does not exist.
    Moreover, we define the \emph{covariant differential} of $T$ as
    \[\nabla T(Y_1,\cdots,Y_r,X)=\nabla_XT(Y_1,\cdots,Y_r).\]
\end{defn}

One can check $\nabla T$ is indeed tensorial, and satisfies Leibniz rule, that is
\[\nabla(T\otimes S)=(\nabla T)\otimes S+T\otimes(\nabla S).\]

We look at covariant derivative locally.
Let
\[T=T_{i_1\cdots i_r}\d{x^{i_1}}\otimes\cdots\otimes\d{x^{i_r}},\]
and write
\[\nabla_{\partial_k}T=T_{i_1\cdots i_r;k}\d{x^{i_1}}\otimes\cdots\otimes\d{x^{i_r}}.\]
Let's compute the coefficient. First we compute
\[\nabla_{\partial_i}\d{x^j}=a_k\d{x^k},\]
then
\begin{align*}
    a_k&=\nabla_{\partial_i}\d{x^j}(\partial_k)\\
    &=-\d{x^j}(\nabla_{\partial_i}\partial_k)\\
    &=-\d{x^j}(\Gamma^l_{ik}\partial_l)\\
    &=-\Gamma^j_{ik},
\end{align*}
that is,
\[\nabla_{\partial_i}d{x^j}=-\Gamma^j_{ik}d{x^k}.\]
Using this, we can compute
\begin{align*}
    &\nabla_{\partial_k}(T_{i_1\cdots i_r}\d{x^{i_1}}\otimes\cdots\otimes\d{x^{i_r}})\\
    =&\partial_kT_{i_1\cdots i_r}\d{x^{i_1}}\otimes\cdots\otimes\d{x^{i_r}}+T_{i_1\cdots i_r}\sum_{l=1}^r\d{x^{i_1}}\otimes\cdots\otimes(\nabla_{\partial_k}\d{x^{i_j}})\otimes\cdots\otimes\d{x^{i_r}}\\
    =&\partial_kT_{i_1\cdots i_r}\d{x^{i_1}}\otimes\cdots\otimes\d{x^{i_r}}+T_{i_1\cdots i_r}\sum_{l=1}^r\d{x^{i_1}}\otimes\cdots\otimes(-\Gamma^{i_j}_{kp}\d{x^p})\otimes\cdots\otimes\d{x^{i_r}}\\
    =&\partial_kT_{i_1\cdots i_r}\d{x^{i_1}}\otimes\cdots\otimes\d{x^{i_r}}-\sum_{l=1}^rT_{i_1\cdots i_{l-1}pi_{l+1}\cdots i_r}\Gamma^p_{ki_l}\d{x^{i_1}}\otimes\cdots\otimes\d{x^{i_r}},
\end{align*}
hence we have
\[T_{i_1\cdots i_r;k}=\partial_kT_{i_1\cdots i_r}-\sum_{l=1}^rT_{i_1\cdots i_{l-1}pi_{l+1}\cdots i_r}\Gamma^p_{ki_l}.\]

Moreover, we can consider second covariant derivative.
\begin{symb}
    We use the symbol $\nabla^2_{X,Y}T$ to denote the tensor 
    \[\nabla^2_{X,Y}T(Y_1,\cdots,Y_r)=\nabla(\nabla T)(Y_1,\cdots,Y_r,Y,X).\]
\end{symb}

\begin{prop}
    We have
    \begin{equation}
        \nabla^2_{X,Y}T=\nabla_X(\nabla_YT)-\nabla_{\nabla_XY}T.\label{2nd cov}
    \end{equation}
\end{prop}
\begin{proof}
    We have
    \begin{align*}
        &\nabla^2_{X,Y}T(Y_1,\cdots,Y_r)\\
        =&\nabla(\nabla T)(Y_1,\cdots,Y_r,Y,X)\\
        =&\nabla_X(\nabla T)(Y_1,\cdots,Y_r,Y)\\
        =&X(\nabla T)(Y_1,\cdots,Y_r,Y)-\sum_{i=1}^r(\nabla T)(Y_1,\cdots,\nabla_XY_i,\cdots,Y_r,Y)\\
        &-(\nabla T)(Y_1,\cdots,Y_r,\nabla_XY)\\
        =&X(\nabla_YT)(Y_1,\cdots,Y_r)-\sum_{i=1}^r(\nabla_YT)(Y_1,\cdots,\nabla_XY_i,\cdots,Y_r)\\
        &-(\nabla_{\nabla_XY}T)(Y_1,\cdots,Y_r)\\
        =&\nabla_X(\nabla_YT)(Y_1,\cdots,Y_r)-(\nabla_{\nabla_XY}T)(Y_1,\cdots,Y_r).\qedhere
    \end{align*}
\end{proof}

Next we discuss curvature of tensors.

\begin{defn}[Curvature operator]\index{curvature!tensor operator}
    Let $X,Y$ be vector fields, define an endomorphism $R(X,Y)$ of $(0,r)$-tensors as
    \[R(X,Y)T=\nabla_Y\nabla_XT-\nabla_X\nabla_YT+\nabla_{[X,Y]}T.\]
\end{defn}

\begin{prop}[Ricci identity]\label{Ricci identity}\index{Ricci identity}
    For a $(0,r)$-tensor $T$ and vector fields $X,Y$, we have
    \begin{align*}
        (\nabla^2_{Y,X}-\nabla^2_{X,Y})T=R(X,Y)T=-\sum_{i=1}^nT(Y_1,\cdots,R(X,Y)Y_i,\cdots,Y_r)
    \end{align*}
\end{prop}
\begin{proof}
    Using equation~\ref{2nd cov}, we have
    \begin{align*}
        (\nabla^2_{Y,X}-\nabla^2_{X,Y})T&=\nabla_Y\nabla_XT-\nabla_{\nabla_YX}T-\nabla_X\nabla_YT+\nabla_{\nabla_XY}T\\
        &=\nabla_Y\nabla_XT-\nabla_X\nabla_YT+\nabla_{(\nabla_XY-\nabla_YX)}T\\
        &=\nabla_Y\nabla_XT-\nabla_X\nabla_YT+\nabla_{[X,Y]}T\quad\text{(torsion-freeness)}\\
        &=R(X,Y)T.
    \end{align*}
    The latter equality holds since $R(X,Y)$ satisfies Leibniz law, then we have
    \begin{align*}
        (R(X,Y)T)(Y_1,\cdots,Y_r)&=R(X,Y)T(Y_1,\cdots,Y_r)-\sum_{i=1}^nT(Y_1,\cdots,R(X,Y)Y_i,\cdots,Y_r)\\
        &=-\sum_{i=1}^nT(Y_1,\cdots,R(X,Y)Y_i,\cdots,Y_r),
    \end{align*}
    the second equality holds since $T(Y_1,\cdots,Y_r)$ is a function.
\end{proof}

Now we introduce some differential operators.

\begin{defn}\index{Hessian}
    Let $(M,g)$ be a Riemannian manifold, $f\in C^\infty(M)$.
    Define the \emph{Hessian} of $f$ to be a $(0,2)$-tensor $\nabla^2f$.
    Ricci identity shows $\nabla^2f$ is symmetric, hence we define its trace to be the \emph{Laplacian} of $f$, denoted by $\Delta f$.
\end{defn}

\begin{prop}
    Let $(M,g)$ be a Riemannian manifold, $f\in C^\infty(M)$.
    Then we have
    \[\nabla^2f(X,Y)=Y(Xf)-(\nabla_YX)(f).\]
\end{prop}
\begin{proof}
    This is just the equation~\ref{2nd cov}.
\end{proof}

\begin{defn}\index{divergence}
    Let $M$ be a differentiable manifold, $X\in\mathfrak{X}(M)$.
    Then we define the \emph{divergence} of $X$ to be $\tr(Y\mapsto\nabla_YX)$.
\end{defn}

\begin{thm}[Divergence theorem]\index{theorem!divergence}
    Let $\Omega$ be a region in $\mathbb{R}^n$ with $\Sigma:=\partial\Omega$ smooth.
    Let $X\in\mathfrak{X}(\Omega)$, then we have
    \[\int_\Sigma\langle X,N\rangle\d{A}=\int_\Omega\div{X}\d\Vol,\]
    where $N$ is the outward unit normal vector field of $\Sigma$.
\end{thm}
Please refer to any book on calculus for the proof.